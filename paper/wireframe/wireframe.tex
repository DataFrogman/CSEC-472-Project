\documentclass[10pt,twocolumn,letterpaper]{article}

\usepackage[english]{babel}
\usepackage[utf8x]{inputenc}
\usepackage[T1]{fontenc}

\usepackage[a4paper,top=3cm,bottom=2cm,left=3cm,right=3cm,marginparwidth=1.75cm]{geometry}

\usepackage{bibentry}
\usepackage[hyphens]{url}
\usepackage[numbib]{tocbibind}
\usepackage{amsmath}
\usepackage{graphicx}
\usepackage[colorinlistoftodos]{todonotes}
\usepackage[colorlinks=true, allcolors=blue]{hyperref}
\setlength{\marginparwidth}{2cm}

\include{references.bib}

\title{
		\vspace{-1in} 	
		\usefont{OT1}{bch}{b}{n}
		\normalfont \normalsize \textsc{CSEC-472 Authentication Paper} \\ [14pt]
		\huge A Security Analysis of the Lenel BlueDiamond \\
}

\usepackage{authblk}

\author[1]{Quintin Walters}
\author[1]{Joshua Niemann}
\author[1]{Connor Leavesley}
\author[1]{Daniel Capps}
\author[1]{Jacob Ruud}

\affil[1]{\small{Computing Security, Rochester Institute of Technology}}

\begin{document}
\maketitle
\selectlanguage{english}
\begin{abstract}

Bluetooth technology has been around since the mid 1990's. It's adoption into consumer technology has made it a mainstream protocol for those developing compatible hardware. Unfortunately, weak security in it's underlying design has also made it a target for multiple different forms of attack. This can be especially dangerous when the compatible device is used for access control. In this paper we study a specific subset of devices from Lenel's new BlueDiamond lineup, testing a variety of methods that have proven affective on similarly designed hardware. These methods included assessments of physical hardware, firmware, and connected BlueDiamond mobile app for vulnerabilities and could result in a breach of access control.

\end{abstract} 

\section{Keywords}

Bluetooth

\section{Introduction}

In the last couple of years, the physical security community has seen an abundace of standard mechanical locks being replaced with \"smart locks\" designed to make accessing secure areas simpler and safer. Only to discover that these smart locks can be opened with just as little effort as picking a standard lock requires. In response to this, corporations have been rolling out revision after revision of their smart lock technology to try and combat the evolution of access control bypass.

Our research revolves around the latest in smart lock technology from Lenels BlueDiamond range. We will use a variety of previously proven methods to attempt to make an unauthorized bypass of the access control upheld be these devices. Methods including physical testing of the wiegand hardware within the device, code review of the device firmware, and analysis of the connection between the device and compatible mobile app for potential vulnerabilities.

Smart locks are the future of physical access control technology, and research from us and other likeminded individuals will ensure that the smart locks deployed in the future are free from as many known issues as possible.

\section{Background \& Significance}
The purpose of this research is to test the security properties of the Lenel Blue Diamond Readers and to uncover different vulnerabilities in them so that they can be disclosed properly and eventually patched. Lenel has over 1 million Blue Diamond readers deployed in multiple industries, the most notable one being real-estate \cite{lenelbluediamondwebsite}. This study is important because the security of these locks are very important because if someone unauthorized is able to enter through these locked doors, everything on the other side would be compromised including important data, objects, and even people. Things which if they were compromised would result in extreme loss. For our research the major issues we found fall into three categories, the firmware used on the lock, the physical aspect of the lock, and the usage of the mobile app and bluetooth communication. We plan to conduct our research by using trusted sources on the subject and testing their conclusions on Lenel Blue Diamond readers. We plan to conduct our research by using trusted sources on the subject and testing their conclusions on Lenel Blue Diamond readers. Our research has it bounds inside the Lenel Blue Diamond readers mobile app and lock itself.

\section{Related Work}

\begin{itemize}
    \item In Finding Vulnerabilities in IoT Devices: Ethical Hacking of Electronic Locks, Robberts and Toft use GATT attacks to Man-in-the-Middle an application and a lock to conduct Bluetooth attacks. Not much digging was done in the way of breaking the cryptographic protocols. 
    \item In Security Vulnerabilities of Bluetooth Low Energy Technology (BLE), O'sullivan found that the communication protocols in BLE were extremely vulnerable to attack.
    \item There is a lot of focus attacking the GATT and GAP elements of BLE. Some research indicates that the Short Term and Long Term keys are brute forceable, but it has yet to be seen. 
\end{itemize}

\section{Research Design \& Methods}
To provide a well rounded analysis of the device there are three primary areas of research.  First, attacks against the physical security of the device.  Second, analysis of the firmware for vulnerabilities.  Finally, testing the application and Bluetooth implementation.

The phyisical security of the lock, particularly the magnetic antitamper switch, is what protects it from well known attacks against the Wiegand protocol.  If the physical security can be bypassed than attackers could take advantage of these other attacks to bypass the lock entirely.

Firmware analysis is necessary to discover underlying vulnerabilities.  The vulnerabilities discovered here may include hardcoded credentials, service backdoors, and poorly implemented authentication methods.  These vulnerabilites would not be obvious or easily detectible without dissecting the firmware.

The application and Bluetooth implementation are the primary means which users will be interacting with the device.  Any vulnerabilities here will be exposed to outside attackers on a daily basis.  Any attacker within 30ft of the device can snoop on the Bluetooth traffic.  The application is also downloadable on the Google Play Store for free, any vulnerabilities in it would be easily accessible to an attacker.

\subsection{Firmware Analysis}
\begin{itemize}
    \item Search for hard-coded credentials.
    \item Search for vulnerabilities such as buffer overflows, default certificates, debugging services, open ports, etc.
    \item Search for improperly implemented authentication and cryptographic methods.
    \item Search for any unpatched vulnerabilities in dependencies and base code such as BlueBorne.
\end{itemize}

Firmware analysis is a solid foundation for the research to build upon.  The firmware is the software that runs on and controls the device as a whole, vulnerabilities evident in it reduce the security of the device as a whole and can lead to vulnerabilities in other areas.  Firmware analysis is performed with automated tools at first, followed by manual analysis to confirm any discoveries.  A final pass through with manual analysis is done to spot any missed or overlooked vulnerabilities.  The methods of analysis will vary, depending on if the firmware is supplied or if it must be extracted it from the device.  If the firmware is provided then analysis can be performed directly, otherwise extraction must be performed and the compiled firmware will be analyzed.

\subsection{Physical Analysis}
\begin{itemize}
    \item Search the lock for potential physical bypass attacks.
    \item Search for issues with the magnetic anti-tamper switch.
    \item Search the lock for easy access to Wiegand protocol leads.
    \item Search for potential areas of destructive entry.
\end{itemize}

Enterprise locks often have many potential types of issues from the perspective of a physical attack.  The authors expect to find potential implementation errors in areas such as the configuration of the anti-tamper switch, the location of the connectors to the Wiegand network, or access to the locking motor from within the device.  In order to detect these issues, the authors will analyze the lock from the perspective of a typical wall-mounted installation.  From there, the authors will look into various ways to both non-destructively, and destructively trigger an unlock.  In a physical test, a vulnerability can take a lot of different forms.  

\subsection{Bluetooth and Application Analysis}
\begin{itemize}
    \item Perform Generic Attribute Protocol (GATT) attacks to gather data about the device and conduct Man-in-the-Middle (MITM) attacks. 
    \item Perform known cryptographic attacks against Bluetooth Low Energy (BLE) to decrypt intercepted traffic, perform replay attacks and command injection.
    \item Search for hard-coded credentials, buffer overflows, and other application vulnerabilities. 
    \item Search for abusable edge cases in the authentication scheme.
\end{itemize}

BLE is a major part of this research. BLE is a Bluetooth protocol variant designed for low energy operations and such has a limited set of computing resources available. BLE does not need manual pairing to a device, meaning an application with the correct authentication data can pair automatically. Automated tools will gather preliminary data about the device by examining the GATT and Generic Access Profile (GAP) . The tool will copy he GATT over to an intermediary device and MITM all communications between the device and the application. With the device communication being intercepted, an Injection Free Attack can force a re-pairing and the intermediary device will intercept the Short Term Key. With the Short Term Key,  An attacker can derive the Long Term Key. They can then use the Long Term Key to decrypt communications and send commands on the behalf of the application. Automated tools will scan the application to discover hard-coded credentials and vulnerabilities. Finally, edge cases in the authentication scheme will be manually explored to see if an attacker could exploit permissions.


\section{Preliminary Suppositions \& Implications}
Just because you don't have to actually conduct the study and analyze the results, doesn't mean you can skip talking about the analytical process and potential implications. The purpose of this section is to argue how and in what ways you believe your research will refine, revise, or extend existing knowledge in the subject area under investigation. Depending on the aims and objectives of your study, describe how the anticipated results will impact future scholarly research, theory, practice, forms of interventions, or policymaking. Note that such discussions may have either substantive [a potential new policy], theoretical [a potential new understanding], or methodological [a potential new way of analyzing] significance.
 
When thinking about the potential implications of your study, ask the following questions:
\begin{itemize}
    \item What might the results mean in regards to challenging the theoretical framework and underlying assumptions that support the study?
    \item What suggestions for subsequent research could arise from the potential outcomes of the study?
    \item What will the results mean to practitioners in the natural settings of their workplace?
    \item Will the results influence programs, methods, and/or forms of intervention?
    \item How might the results contribute to the solution of social, economic, or other types of problems?
    \item Will the results influence policy decisions?
    \item In what way do individuals or groups benefit should your study be pursued?
    \item What will be improved or changed as a result of the proposed research?
    \item How will the results of the study be implemented and what innovations or transformative insights could emerge from the process of implementation?
\end{itemize}

\textbf{NOTE:} This section should not delve into idle speculation, opinion, or be formulated on the basis of unclear evidence. The purpose is to reflect upon gaps or understudied areas of the current literature and describe how your proposed research contributes to a new understanding of the research problem should the study be implemented as designed.

\section{Conclusions}
The conclusion reiterates the importance or significance of your paper and provides a brief summary of the entire study. This section should be only one or two paragraphs long, emphasizing why the research problem is worth investigating, why your research study is unique, and how it should advance existing knowledge.

Someone reading this section should come away with an understanding of:
\begin{itemize}
    \item Why the study should be done,
    \item The specific purpose of the study and the research questions it attempts to answer,
    \item The decision to why the research design and methods used where chosen over other options,
    \item The potential implications emerging from your proposed study of the research problem, and
    \item A sense of how your study fits within the broader scholarship about the research problem.
\end{itemize}

\section{Acknowledgements}
We would like to thank Lenel for access to their BlueDiamond lock and permission to conduct a test of its security. 

\bibliographystyle{ieee}
\bibliography{references}

\bibentry{lenelbluediamondwebsite}

\end{document}
