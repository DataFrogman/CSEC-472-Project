\section{Summary 9}

\noindent
\subsection{Title}

\subsubsection{Group Member}

\noindent
Daniel Capps

\noindent
\subsubsection{Citation}

\Urlmuskip=0mu plus 1mu\relax
%remove % to show citation, change the name to your entry
%\bibentry{filizzola2018security}

\subsubsection{Main Idea}

\noindent
Main Idea

\subsubsection{Theory}

\noindent
Theory

\subsubsection{Method}

\noindent
Method

\subsubsection{Findings}

\noindent
Findings

\subsubsection{Future Directions}

\noindent
Future Directions 

\Urlmuskip=0mu plus 1mu\relax

\noindent
\subsection{Classifying RFID attacks and defenses}

\subsubsection{Group Member}

\noindent
Connor Leavesley

\noindent
\subsubsection{Citation}

\Urlmuskip=0mu plus 1mu\relax
%remove % to show citation, change the name to your entry
\bibentry{mitrokotsa09}

\subsubsection{Main Idea}

\noindent
The authors classify existing RFID attacks and discuss possible countermeasures.

\subsubsection{Theory}

\noindent
The authors are using Classification Theory. RFID attacks are being split into subgroups based on similarities and differences. 

\subsubsection{Method}

\noindent
The authors classify attacks based on the layer where the attack takes place: physical layer, network-transport layer, application layer, strategic layer, and multilayer. They discuss the attacks and offer possible solutions to the issues. 

\subsubsection{Findings}

\noindent
The authors found that most of the attacks possible against RFID were at the physical layer via passive and active interference methods such as sending KILL commands, replay attacks, or destruction of the tags. Multiplayer attacks were only cryptobased attacks. Network-Transport layer had the tag attacks that are more common, such as cloning and spoofing. 

\subsubsection{Future Directions}

\noindent
The authors offer no indication of a future direction. 

\Urlmuskip=0mu plus 1mu\relax

\noindent
\subsection{Title}

\subsubsection{Group Member}

\noindent
Joshua Niemann

\noindent
\subsubsection{Citation}

\Urlmuskip=0mu plus 1mu\relax
%remove % to show citation, change the name to your entry
%\bibentry{filizzola2018security}

\subsubsection{Main Idea}

\noindent
Main Idea

\subsubsection{Theory}

\noindent
Theory

\subsubsection{Method}

\noindent
Method

\subsubsection{Findings}

\noindent
Findings

\subsubsection{Future Directions}

\noindent
Future Directions 

\Urlmuskip=0mu plus 1mu\relax

\noindent
\subsection{Title}

\subsubsection{Group Member}

\noindent
Jacob Ruud

\noindent
\subsubsection{Citation}

\Urlmuskip=0mu plus 1mu\relax
%remove % to show citation, change the name to your entry
%\bibentry{filizzola2018security}

\subsubsection{Main Idea}

\noindent
Main Idea

\subsubsection{Theory}

\noindent
Theory

\subsubsection{Method}

\noindent
Method

\subsubsection{Findings}

\noindent
Findings

\subsubsection{Future Directions}

\noindent
Future Directions 

\Urlmuskip=0mu plus 1mu\relax

\noindent
\subsection{Securing RFID systems by detecting tag cloning}

\subsubsection{Group Member}

\noindent
Quintin Walters

\noindent
\subsubsection{Citation}

\Urlmuskip=0mu plus 1mu\relax
%remove % to show citation, change the name to your entry
\bibentry{lehtonen09}

\subsubsection{Main Idea}

\noindent
Cloning RFID cards bypasses the security that they provide.  Detecting the use of cloning could increase the security of an RFID card based solution.

\subsubsection{Theory}

\noindent
Game theory, if cards can be cloned then the security of the system is completely compromised.

\subsubsection{Method}

\noindent
The researchers developed a system that would record the synchronized secret on the card.  It would then check if the secret is identical to the most recent secret written to it.  They used EPC Class-1 Gen-2 tags from UPM Raflatac and a CAEN A828EU UHF reader.  Using 100 reads they attempted to use this method to detect cloning.

\subsubsection{Findings}

\noindent
The authors found that a triggered alarm would be valid 50\% of the time and that a 99.15\% of cloned cards would trigger an alarm.  This has a high false positive rate but a very low false negative rate.

\subsubsection{Future Directions}

\noindent
The authors do not present possible future directions, however, a possible future direction is to test this implementation with more card types and readers along with testing on an active audience to get more accurate results. 

\Urlmuskip=0mu plus 1mu\relax
\pagebreak
