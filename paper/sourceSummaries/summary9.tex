\section{Summary 9}

\noindent
\subsection{{P}rox-{R}{B}{A}{C}: a proximity-based spatially aware {R}{B}{A}{C}}

\subsubsection{Group Member}

\noindent
Daniel Capps

\noindent
\subsubsection{Citation}

\Urlmuskip=0mu plus 1mu\relax

\bibentry{kirkpatrick2011prox}

\subsubsection{Main Idea}

\noindent
The authors extend spatially aware role-based access control (RBAC) by defining proximity-based RBAC (Prox-RBAC).  In the author’s method the access control decisions are made by considering the primary user’s location and the location of every other user in the system. The authors introduce a spatial model and proximity constraints. The authors also define the Prox-RBAC language which is used to specify policy constraints, introduce their enforcement architecture, and present an informal security analysis. 

\subsubsection{Theory}

\noindent
The authors use Game Theory because they have to prove that their implementation is secure against different attack vectors.

\subsubsection{Method}

\noindent
The authors implemented a Prox-RBAC prototype in which they use a centralized authorization server with different clients. In their implementation the clients could be either stationary or mobile devices but would be required to provide a high-integrity proof of location. The authors Prox-RBAC prototype is built on the interaction between four principals. The authorization server (AS) is the centralized server that acts as the policy decision point (PDP). The AS maintains a mapping of all users’ locations and all access control policies. Each user (subject) s will have an id, a password, and a proximity-connection device that is used for generating the proof of location. This device has a secret and a certificate signed by the AS that allows the device to prove they know the secret without the server knowing the secret.

\subsubsection{Findings}

\noindent
The authors developed a prototype of Prox-RBAC to measure the performance of the cryptographic protocols and enforcement algorithms. The Prove construct uses the Feige-Fiat-Shamir identification protocol which uses a zero-knowledge proof. The Commit and Open primitives use Pedersen commitment. For authentication the authors use a salted hash of a password. The Hash, Enck, Encpk(c), and Signk primitives use SHA-256, AES-256, 1024-bit RSA, and SHA-1 with DSA in order. After 500 tests, the Pedersen commitment (37.9 ms) and RSA (35.9 ms) were the most expensive cryptographic operations while the rest with the exception of DSA signature (9.8 ms) required less than 1 ms to complete on average. 

\subsubsection{Future Directions}

\noindent
Is there a better implementation of Feige-Fiat-Shamir and Pedersen that doesn’t use the BigInteger java library so said protocols can be used in different configurations? Can we use the results of this paper to construct a usable and efficient proximity-based RBAC system?

\Urlmuskip=0mu plus 1mu\relax


\noindent
\subsection{Classifying RFID attacks and defenses}

\subsubsection{Group Member}

\noindent
Connor Leavesley

\noindent
\subsubsection{Citation}

\Urlmuskip=0mu plus 1mu\relax
%remove % to show citation, change the name to your entry
\bibentry{mitrokotsa09}

\subsubsection{Main Idea}

\noindent
The authors classify existing RFID attacks and discuss possible countermeasures.

\subsubsection{Theory}

\noindent
The authors are using Classification Theory. RFID attacks are being split into subgroups based on similarities and differences. 

\subsubsection{Method}

\noindent
The authors classify attacks based on the layer where the attack takes place: physical layer, network-transport layer, application layer, strategic layer, and multilayer. They discuss the attacks and offer possible solutions to the issues. 

\subsubsection{Findings}

\noindent
The authors found that most of the attacks possible against RFID were at the physical layer via passive and active interference methods such as sending KILL commands, replay attacks, or destruction of the tags. Multiplayer attacks were only cryptobased attacks. Network-Transport layer had the tag attacks that are more common, such as cloning and spoofing. 

\subsubsection{Future Directions}

\noindent
The authors offer no indication of a future direction. 

\Urlmuskip=0mu plus 1mu\relax

\noindent
\subsection{Title}

\subsubsection{Group Member}

\noindent
Joshua Niemann

\noindent
\subsubsection{Citation}

\Urlmuskip=0mu plus 1mu\relax
%remove % to show citation, change the name to your entry
%\bibentry{filizzola2018security}

\subsubsection{Main Idea}

\noindent
Main Idea

\subsubsection{Theory}

\noindent
Theory

\subsubsection{Method}

\noindent
Method

\subsubsection{Findings}

\noindent
Findings

\subsubsection{Future Directions}

\noindent
Future Directions 

\Urlmuskip=0mu plus 1mu\relax

\noindent
\subsection{{P}ractical {B}luetooth {T}raffic {S}niffing: {S}ystems and {P}rivacy {I}mplications}

\subsubsection{Group Member}

\noindent
Jacob Ruud

\noindent
\subsubsection{Citation}

\Urlmuskip=0mu plus 1mu\relax
%remove % to show citation, change the name to your entry
\bibentry{albazrqaoe2016practical}

\subsubsection{Main Idea}

\noindent
The authors discuss privacy concerns that exist with the prevalence of bluetooth sniffing in todays iot landscape. During their discussion they propose a device of their own creation that can sniff bluetooth traffic effectively using commercially available bluetooth-compliant radios.

\subsubsection{Theory}

\noindent
Game theory is used in this paper as a method of measuring success of the new reader design; Where a win is represented by the readers ability to sniff blbluetooth traffic.

\subsubsection{Method}

\noindent
The authors improve upon the design of the Ubertooth reciever by modifying the source code to add support for adaptive hop selection and run-time clock synchronization. They also found that the firmware implementation of real-time frequency hopping needed updating to achieve the results they were aiming for.

\subsubsection{Findings}

\noindent
The authors found that their ubertooth implementation is significantly more efficiant than any previous implementation, but does not accommadate for low energy traffic. Therefore their algorithm would need to be modified in order to sniff Bluetooth Low Energy traffic as effectively as they were able to sniff normal bluetooth traffic.

\subsubsection{Future Directions}

\noindent
The authors point out that their bluetooth sniffer implementation is not configured for BLE traffic. They outline in their paper that some modifications would need to be made to their source code in order for this functionality to be added.

\Urlmuskip=0mu plus 1mu\relax

\noindent
\subsection{Securing RFID systems by detecting tag cloning}

\subsubsection{Group Member}

\noindent
Quintin Walters

\noindent
\subsubsection{Citation}

\Urlmuskip=0mu plus 1mu\relax
%remove % to show citation, change the name to your entry
\bibentry{lehtonen09}

\subsubsection{Main Idea}

\noindent
Cloning RFID cards bypasses the security that they provide.  Detecting the use of cloning could increase the security of an RFID card based solution.

\subsubsection{Theory}

\noindent
Game theory, if cards can be cloned then the security of the system is completely compromised.

\subsubsection{Method}

\noindent
The researchers developed a system that would record the synchronized secret on the card.  It would then check if the secret is identical to the most recent secret written to it.  They used EPC Class-1 Gen-2 tags from UPM Raflatac and a CAEN A828EU UHF reader.  Using 100 reads they attempted to use this method to detect cloning.

\subsubsection{Findings}

\noindent
The authors found that a triggered alarm would be valid 50\% of the time and that a 99.15\% of cloned cards would trigger an alarm.  This has a high false positive rate but a very low false negative rate.

\subsubsection{Future Directions}

\noindent
The authors do not present possible future directions, however, a possible future direction is to test this implementation with more card types and readers along with testing on an active audience to get more accurate results. 

\Urlmuskip=0mu plus 1mu\relax
\pagebreak
