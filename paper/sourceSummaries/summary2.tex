\section{Summary 2}

\noindent
\subsection{{A}nalyzing the {S}ecurity of {B}luetooth {L}ow {E}nergy}

\subsubsection{Group Member}

\noindent
Connor Leavesley

\noindent
\subsubsection{Citation}

\Urlmuskip=0mu plus 1mu\relax
%remove % to show citation, change the name to your entry

\bibentry{sevier2019securityofble}


\subsubsection{Main Idea}

\noindent
The authors aim to explain how Bluetooth LE protocol works and the cryptographic weaknesses in the protocol.


\subsubsection{Theory}

\noindent
This paper is applying Game Theory. A successful cryptographic attack is a loss for the security of the protocol. 

\subsubsection{Method}

\noindent
The authors first sniffed the Bluetooth traffic with a Ubertooth using the BlueZ Bluetooth driver and associated Ubertooth drivers. Using the Blueooth handshake, the authors used Crackle to crack the Temporary Key due to its restricted key space. The Temporary Key was then used to gain access to the Long Term Key (LTK). The LTK was then used to decrypt any future communication traffic in Wireshark. 

\subsubsection{Findings}

\noindent
The authors found that the keyspace of the Temporary Key is very restricted, allowing for a very quick brute force attack. They also found that Bluetooth Low Energy was susceptible to a number of attacks due to the low power requirements: denial of service and replay attacks. It was also found that Ubertooth struggled to capture a complete pairing event. The authors suggest that the Ubertooth should be as close as possible to the source transceiver to mitigate this issue. 

\subsubsection{Future Directions}

\noindent
Many vendors likely do not implement the Bluetooth stack correctly. These vendors may also use the same stack across multiple devices. Areas of further research should focus on individual devices from the same vendor to attempt to find vulnerabilities that affect entire product lines. 

\noindent
\subsection{{W}iegand {P}rotocol {A}ccess: {A} {D}ecade of {D}ecryption}

\subsubsection{Group Member}

\noindent
Quintin Walters

\noindent
\subsubsection{Citation}

\Urlmuskip=0mu plus 1mu\relax
%remove % to show citation, change the name to your entry
\bibentry{chung2017wiegand}

\subsubsection{Main Idea}

\noindent
Brandon Chung covers what the Wiegand Protocol is, the historic vulnerabilities and hacks, what vulnerabilities still exist, and how to protect yourself against them.  He spends a large amount of time on the historic attacks because most of them are still applicable, the protocol has not been hardened against them and as a result it is still very easy to exploit.

\subsubsection{Theory}

\noindent
Chung applies Game Theory in his work, he treats the security of the Wiegand Protocol as a zero sum game in which any method of bypass is a loss for the defenders.  He highlights historic vulnerabilities of the protocol that are still in existence to reinforce this belief.

\subsubsection{Method}

\noindent
The author primarily presents the findings of others, he does little original research of his own.  However, he does use an Arduino device to attack an unnamed Wiegand RFID reader.  He connects his Arduino device to the Wiegand DATA 0 and DATA 1 wires, then he uses monkeyboard's "Wiegand Protocol Library for Arduino" to verify that the inherent vulnerabilities in the protocol still exist.  Chung provides instructions and sample code for readers to attempt this on their own devices. This attack is the basis of the attacks done by Bernard Mehl (2015) and Zac Franken (2007), two attacks that Chung wrote in depth about.

\subsubsection{Findings}

\noindent
Brandon Chung found that Wiegand devices are still vulnerable to decade old attacks.  These attacks have been extensively documented and Chung duplicated the early stages of them to prove that they would still work.  Using an arduino device, or similar microcontroller, an attacker can intercept and then duplicate the signals sent by a Wiegand device to the control server. This attack can capture and repeat and authorized card without needing to physically duplicate the card.

\subsubsection{Future Directions}

\noindent
Chung lays out multiple methods for future implementation to secure Wiegand devices.  He recommends that the protocol be adapted to allow for encrypted keycards and the rejection of keycards that are not properly encrypted, he also recommends that the actual readers implement hardware methods to detect when the device has been tampered with and to report that tampering immediately to the controller.  Further upgrades include remote firmware detection and updates, Wiegand devices typically are not able to be updated without direct physical contact which disincentivizes updating the readers unless absolutely necessary.

\Urlmuskip=0mu plus 1mu\relax

\noindent
\subsection{{S}urvey on {V}arious {D}oor {L}ock {A}ccess {C}ontrol {M}echanisms}

\noindent
\subsubsection{Group Member}

\noindent
Joshua Niemann

\noindent
\subsubsection{Citation}

\Urlmuskip=0mu plus 1mu\relax
\bibentry{matthewrs2017surveydoorlock}

\subsubsection{Main Idea}

\noindent
The authors compare different types of door lock authentication measures and their overall security.

\subsubsection{Theory}

\noindent
The authors are applying Game Theory by directly comparing each individual authentication method.

\subsubsection{Method}

\noindent
The authors directly compare different authentication methods using different factors using characteristics that would matter to a user in addition to the overall security of the system.  For user factors, factors considered include include battery life, ease of use and what happens if a credential is misplaced, stolen or forgotten.  As for security, factors considered included the a user passing off a credential to an unauthorized party, the ability to spoof a credential, and the ease of bypass for a mechanical system.

\subsubsection{Findings}

\noindent
The authors find that no one current system could be considered the most secure, and that each system has individual strengths.  As each method does better in different ways, emphasis should be instead placed on ensuring the use of the lock that is most suited for the use case in which it is placed.  

\subsubsection{Future Directions}

\noindent
The authors suggest using the knowlege of authentication system problems in order to build a new system, taking in mind the strengths and weaknesses for each category of authentication.

\subsection{{B}luetooth {L}ow {E}nergy and {S}martphones for {P}roximity-{B}ased  {A}utomatic {D}oor {L}ocks}

\subsubsection{Group Member}

\noindent
Daniel Capps

\noindent
\subsubsection{Citation}

\Urlmuskip=0mu plus 1mu\relax
%remove % to show citation, change the name to your entry
\bibentry{andersson2014bluetooth}

\subsubsection{Main Idea}

\noindent
The authors attempt to evaluate Bluetooth Low Energy as a technology by focusing on its use in door locks that automatically unlock based on the proximity of a smartphone.


\subsubsection{Theory}

\noindent
Game theory. The authors try to determine ways to use the Proximity-based Automatic Door Locks with Bluetooth Low Energy without causing any adversaries to be able to exploit the locks. 

\subsubsection{Method}

\noindent
The author’s main method for testing their hypothesis had two halfs. The first half was to implement an application for iOS that would be able to unlock/lock the door lock, and the second half was to measure the Received Signal Strength Indicator (RSSI) between the phone and the door lock in order to ensure that the door would only unlock if the phone was in close enough proximity, and otherwise would be locked.

\subsubsection{Findings}

\noindent
The authors gained a lot of knowledge pertaining to how suitable Bluetooth Low Energy is for automatic door locks, including what restrictions and possibilities exist on the iOS platform for developing Bluetooth Low Energy applications. The authors primarily learned that using this method they determined that they couldn’t differentiate between the two sides of the door the user was located on, that the applications effect on the battery life of the phone was negligible, the connection latency was sufficiently small for use in practice. The author's conclusion is that Bluetooth Low Energy is a suitable technology for proximity-based door locks.


\subsubsection{Future Directions}

\noindent
Can you improve the implemented solution so that there is no limit on how many people can be in close proximity to the lock at once? Is there a better alternative to Bluetooth Low Energy in the case of Proximity-based door locks?


\Urlmuskip=0mu plus 1mu\relax

\pagebreak