\section{Summary 2}

\noindent
\subsection{{A}nalyzing the {S}ecurity of {B}luetooth {L}ow {E}nergy}

\subsubsection{Group Member}

\noindent
Connor Leavesley

\noindent
\subsubsection{Citation}

\Urlmuskip=0mu plus 1mu\relax
%remove % to show citation, change the name to your entry

\bibentry{sevier2019securityofble}


\subsubsection{Main Idea}

\noindent
The authors aim to explain how Bluetooth LE protocol works and the cryptographic weaknesses in the protocol.


\subsubsection{Theory}

\noindent
This paper is applying Game Theory. A successful cryptographic attack is a loss for the security of the protocol. 

\subsubsection{Method}

\noindent
The authors first sniffed the Bluetooth traffic with a Ubertooth using the BlueZ Bluetooth driver and associated Ubertooth drivers. Using the Blueooth handshake, the authors used Crackle to crack the Temporary Key due to its restricted key space. The Temporary Key was then used to gain access to the Long Term Key (LTK). The LTK was then used to decrypt any future communication traffic in Wireshark. 

\subsubsection{Findings}

\noindent
The authors found that the keyspace of the Temporary Key is very restricted, allowing for a very quick brute force attack. They also found that Bluetooth Low Energy was susceptible to a number of attacks due to the low power requirements: denial of service and replay attacks. It was also found that Ubertooth struggled to capture a complete pairing event. The authors suggest that the Ubertooth should be as close as possible to the source transceiver to mitigate this issue. 

\subsubsection{Future Directions}

\noindent
Many vendors likely do not implement the Bluetooth stack correctly. These vendors may also use the same stack across multiple devices. Areas of further research should focus on individual devices from the same vendor to attempt to find vulnerabilities that affect entire product lines. 

\noindent
\subsection{{W}iegand {P}rotocol {A}ccess: {A} {D}ecade of {D}ecryption}

\subsubsection{Group Member}

\noindent
Quintin Walters

\noindent
\subsubsection{Citation}

\Urlmuskip=0mu plus 1mu\relax
%remove % to show citation, change the name to your entry
\bibentry{chung2017wiegand}

\subsubsection{Main Idea}

\noindent
Brandon Chung covers what the Wiegand Protocol is, the historic vulnerabilities and hacks, what vulnerabilities still exist, and how to protect yourself against them.  He spends a large amount of time on the historic attacks because most of them are still applicable, the protocol has not been hardened against them and as a result it is still very easy to exploit.

\subsubsection{Theory}

\noindent
Chung applies Game Theory in his work, he treats the security of the Wiegand Protocol as a zero sum game in which any method of bypass is a loss for the defenders.  He highlights historic vulnerabilities of the protocol that are still in existence to reinforce this belief.

\subsubsection{Method}

\noindent
The author primarily presents the findings of others, he does little original research of his own.  However, he does use an Arduino device to attack an unnamed Wiegand RFID reader.  He connects his Arduino device to the Wiegand DATA 0 and DATA 1 wires, then he uses monkeyboard's "Wiegand Protocol Library for Arduino" to verify that the inherent vulnerabilities in the protocol still exist.  Chung provides instructions and sample code for readers to attempt this on their own devices. This attack is the basis of the attacks done by Bernard Mehl (2015) and Zac Franken (2007), two attacks that Chung wrote in depth about.

\subsubsection{Findings}

\noindent
Brandon Chung found that Wiegand devices are still vulnerable to decade old attacks.  These attacks have been extensively documented and Chung duplicated the early stages of them to prove that they would still work.  Using an arduino device, or similar microcontroller, an attacker can intercept and then duplicate the signals sent by a Wiegand device to the control server. This attack can capture and repeat and authorized card without needing to physically duplicate the card.

\subsubsection{Future Directions}

\noindent
Chung lays out multiple methods for future implementation to secure Wiegand devices.  He recommends that the protocol be adapted to allow for encrypted keycards and the rejection of keycards that are not properly encrypted, he also recommends that the actual readers implement hardware methods to detect when the device has been tampered with and to report that tampering immediately to the controller.  Further upgrades include remote firmware detection and updates, Wiegand devices typically are not able to be updated without direct physical contact which disincentivizes updating the readers unless absolutely necessary.

\Urlmuskip=0mu plus 1mu\relax

\subsection{Security Vulnerabilities of Bluetooth Low Energy Technology (BLE)}

\subsubsection{Group Member}

\noindent
Jacob Ruud

\noindent
\subsubsection{Citation}

\Urlmuskip=0mu plus 1mu\relax
%remove % to show citation, change the name to your entry

\bibentry{osullivan}

\subsubsection{Main Idea}

\noindent
Author Harry O'Sullivan along with mentor Ming Chow attempt to analyze the security of bluetooth low energy technology. They focus specifically on how BLE devices communicate with each other and how the communication between devices could be exploited by attackers.

\subsubsection{Theory}

\noindent
O'sullivan applies Game Theory in his work on BLE devices. Posing as an attacker he and attempts to simulate attack scenarios treating a compromise of sensitive information as a win.

\subsubsection{Method}

\noindent
The author presents three different methods to compromise BLE as part of his research. The first method he attemps is an eavesdropping attack, during which he tries to capture information about a bluetooth host device using a sniffer. The second attack method he outlines is Man-in-the-middle, to explain which he refers to research done by lecturers at Marthandam College. The third attack scenario he describes is denial of service. During his explaination he outlines work done by researchers at University of Utah, as well as an attack technique called fuzzing.

\subsubsection{Findings}

\noindent
The main findings ot this paper present the vulnerabilities of the BD ADDR field which is present in all BLE communication. O'Sullivan found that an attacker could use bluetooth fuzzing to determine the master source of a bluetooth connection and potentially forge a connection to it. Not to mention the fact that BD ADDR's are not globally unique so an attacker could try to spoof their BD ADDR until packets start flowing in.

\subsubsection{Future Directions}

\noindent
The author does not include any future directions at the end of his work. Bluetooth Low Energy has been tested extensively since its inception, and I can only assume that O'Sullivan believes that there is work out there testing every corner or bluetooth low energy. I do not necessarily agree with his thoughts but that is all there is to report with this paper.

\Urlmuskip=0mu plus 1mu\relax

\pagebreak