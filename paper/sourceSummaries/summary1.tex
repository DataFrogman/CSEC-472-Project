\section{Summary 1}

\noindent
\subsection{Security Analysis of Bluetooth Technology}

\subsubsection{Group Member}

\noindent
Quintin Walters

\noindent
\subsubsection{Citation}

\Urlmuskip=0mu plus 1mu\relax
\bibentry{filizzola2018security}

\subsubsection{Main Idea}

\noindent
The authors detail the security methods used in Bluetooth versions 4.X, attempt to show attacks that bypass these methods, and describe ways to harden Bluetooth security against these attacks.  

\subsubsection{Theory}

\noindent
Game Theory is the primary theory being tested in this article.  The authors attack various vulnerabilities in the Bluetooth protocol in order to determine methods to increase the relative security of the protocol.

\subsubsection{Method}

\noindent
The researchers theorized and tested primary attacks against the Bluetooth security model: Active Eavesdropping and Passive Eavesdropping.  Their attacks built upon the works by Da-Zhi Sun et al., Cope et a., Das et al., and Ryan.  The assets used were a Raspberry Pi running Debian, an Ubertooth, TaoTronics TT-BH07 Bluetooth Headphones, a Logitech MX Master Mouse, and a Galazy S7 Edge.  The authors modified existing Bluetooth utilities for their attacks and wrote some scripts of their own.

\subsubsection{Findings}

\noindent
The authors managed to exploit their theorized vulnerabilities successfully.  They found that the JustWorks authentication method used by headsets and headphones are insecure against active eavesdropping attacks with unsophisticated hardware.  The researchers also discovered that, while difficult, passive eavesdropping is still successful against hardware running Bluetooth 4.1 and recommend moving to version 4.2 or greater.  They also found that devices using LE Secure Connections or Secure Simple Pairings are secure against these specific attacks.

\subsubsection{Future Directions}

\noindent
The future directions for Active Eavesdropping is to expand and cover more than JustWorks devices, this would include attacks against mice and keyboards.  The next steps for Passive Easedropping is to use the extrapolated information to decrypt packets for further analysis and to attack other devices like keyboards and medical implants, this can be used to gather sensitive information like passwords and health data.  They also stated that they could combine the two attack types to inject malicious packets or modify existing ones for other attacks against the devices.  Finally, they could also do research on the vulnerabilities in Bluetooth 5.0.

\Urlmuskip=0mu plus 1mu\relax

\subsection{Extracting the Security Features Implemented in a Bluetooth LE Connection}

\noindent
\subsubsection{Group member}
Joshua Niemann

\subsubsection{Citation}

\Urlmuskip=0mu plus 1mu\relax
\bibentry{8622000}

\subsubsection{Main Idea}

\noindent
The authors detail the encryption and authentication standards found in the Bluetooth Low Energy Protocol, from Bluetooth 4.0 to Bluetooth 5.0.  The researchers also look at several BLE devices to analyze their security models.

\subsubsection{Theory}

\noindent
Game Theory is the primary theory being tested in this article.  The authors examine various versions of the Bluetooth protocol to determine which versions have the best security measures in place.  In addition, the authors analyze BLE devices in an effort to learn which devices and device manufacturers have the best security.

\subsubsection{Method}

\noindent
The researchers found that many modern fitness trackers use older versions of Bluetooth low energy.  They also found that these fitness trackers often do not enable optional features that make bluetooth much more secure.  As a result, most of the encryption was trivial to decrypt.  The researchers then created an app that uses the Bluetooth sniff log functionality present in the developer section in Android to extract the security features of a bluetooth low energy device.

\subsubsection{Findings}

\noindent
The authors managed to exploit their theorized vulnerabilities successfully.  They found that the JustWorks authentication method used by headsets and headphones are insecure against active eavesdropping attacks with unsophisticated hardware.  The researchers also discovered that, while difficult, passive eavesdropping is still successful against hardware running Bluetooth 4.1 and recommend moving to version 4.2 or greater.  They also found that devices using LE Secure Connections or Secure Simple Pairings are secure against these specific attacks.

\subsubsection{Future Directions}

\noindent
The authors want to make their app much more usable to the average consumer.  Right now it's very oriented toward technical-minded researchers.

\subsection[Finding Vulnerabilities in IoT Devices: Ethical Hacking of Electronic Locks]{Finding Vulnerabilities in IoT Devices: Ethical \\Hacking of Electronic Locks}

\subsubsection{Group Member}

\noindent
Connor Leavesley

\noindent
\subsubsection{Citation}

\Urlmuskip=0mu plus 1mu\relax
\bibentry{KTH}

\subsubsection{Main Idea}

\noindent
The authors outline how to approach attacking Bluetooth locks. They use well documented attacks to discover vulnerabilities in a Bluetooth low energy lock.

\subsubsection{Theory}

\noindent
Game theory is being utilized to test the security of the lock and see where the vulnerabilities in the lock lie. 

\subsubsection{Method}

\noindent
The authors create a threat rating system following the DREAD model is to accurately represent the threat of the lock being successfully attacked. They make three threat models: unauthorized lock access, avoidance of logging, and denial of service. They use a Bluetooth man in the middle attack to gather information about the lock’s Generic Attribute Profile (GATT), how the application connected to the lock, hosted services, and other characteristics of the lock. Using the GATT, an attacker could use a fake lock hosted on a Arduino board to connect to the application, and then forward all information to the actual lock. Next, the authors test access permission edge cases to see when access could be abused by an attacker to bypass controls. The authors then reversed the app to gather further insight on how the app connected to the lock.

\subsubsection{Findings}

\noindent
Not much was able to be done in the way of breaking into the lock. Some patterns were found in the communication scheme, but nothing to indicate how the application was encrypting the connection. The authors lauched a replay attack, but this was ineffective. Fuzzing was similarly ineffective. If the owner of the lock granted an attacker permission to access it, disconnects their internet, and then the attacker's permission was revoked, the attacker can still access the lock. In the app, a possible encryption key for a database was found, but not pursued.

\subsubsection{Future Directions}

\noindent
The authors would like to see a more structured method developed to analyze the security of Bluetooth devices, as many of the approaches they took were very time consuming. 

\subsection{Security Evaluation and Exploitation of Bluetooth Low Energy Devices}

\subsubsection{Group Member}

\noindent
Connor Leavesley

\noindent
\subsubsection{Citation}

\Urlmuskip=0mu plus 1mu\relax
\bibentry{rose_2017}

\subsubsection{Main Idea}

\noindent
In his thesis, Captian Rose attemps to provide the industry with a wholistic look on the security of the low energy bluetooth protocol. He uses four different methods in an attempt to cover a wide range of conigurations and scenarios to really put this protocol to the test.

\subsubsection{Theory}

\noindent
Game Theory is the primary theory being tested in this article.  Captain Rose proposes multiple different scenarios where the bluetooth protocol is vulnerable in order to better understand how to secure it in the future.

\subsubsection{Method}

\noindent
Captain Rose and his supporters used multiple different methods to test the strength and security of the Bluetooth protocol. First, he used a benchmark open-source range-finding tool to determine the location of Bluetooth devices through a novel distance estimation method, increasing the state-of-the-art device location distance from 50 meters up to 1,000 meters. Second, he evaluated 17 individual Bluetooth Low Energy devices for vulnerabilities in their operating system. Thirdly, he used user behavior analytics to demonstrate how malicious actors can exploit vulnerabilities for unauthorized device access and obtain sensitive information

\subsubsection{Findings}

\noindent
Captain Rose's research revealed that 13 out of 17 (75 percent) of the tested devices contained at least one vulnerability resulting in unauthorized access. However, he did find that countermeasures to BLE attacks already exist and most require minimum implementation and development by manufacturers. More complex mediation techniques exist for issues that are not solved by the initial mediation techniques.

\subsubsection{Future Directions}

\noindent
Since Captain Rose's work was done on an outdated bluetooth protocol (4.1) he mentions that there is room for others or himself to do research on devices implementing the newest protocol. He also recommends research into newly release bluetooth devices such as Lockitron or Schlage. He also states that he did not do any work with the firmware of the devices he tested so there is room for others to continue his work there. Along with that, he states that no research was done on attempting to clone the devices that he was testing, so there is more work to be done there.

\noindent
\subsection{Security Vulnerabilities in Bluetooth Technology as used in IoT}

\subsubsection{Group Member}

\noindent
Daniel Capps

\noindent
\subsubsection{Citation}

\Urlmuskip=0mu plus 1mu\relax
\bibentry{lonzetta2018security}

\subsubsection{Main Idea}

\noindent
This study is about understanding Bluetooth and Iot(Internet of Things) devices. They go in depth on the importance of understanding Bluetooth, the differet attacks possible against Bluetooth and IoT devices, and how to mitigate these attacks using different tequniques.  

\subsubsection{Theory}

\noindent
Game Theory is the primary theory being used in this article.  The authors gathered information about many vulnerabilities and exploits in Bluetooth/Iot and determined different mitigation tequniques for each of them. 

\subsubsection{Method}

\noindent
The researchers explained the inner workings of IoT and Bluetooth and then gathered different information on attacks to Bluetooth and Iot. Then, by using the information they found they determined different risk mitigation tequniques for users of Bluetooth and IoT devices.

\subsubsection{Findings}

\noindent
The authors presented multiple vulnerabilities and attacks against Bluetooth and IoT and risk mitigations to combat them, the most notable of them being more awareness from the user's perspective on security issue with bluetooth. The main attacks showcased here are PIN Cracking, Man-in-the-M(MITM), BlueJacking, BlueBorne, Fuzzing, Reflection/Relay, Backdoor, Denial of Service(DOS). Many more attacks are mention as well.

\subsubsection{Future Directions}

\noindent
Can the analysis tequniques use here be used for any other wireless standard? Could there be an attack on BLE that has to do with power consumtion? What methods can we use to more properly inform users of the risks of IoT and Bluetooth so that they may mitigate said risks?

\Urlmuskip=0mu plus 1mu\relax

\pagebreak
