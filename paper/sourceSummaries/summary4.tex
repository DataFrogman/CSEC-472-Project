\section{Summary 4}

\noindent
\subsection{{O}n {P}rivacy and {S}ecurity {C}hallenges in {S}mart {C}onnected {H}omes}

\subsubsection{Group Member}

\noindent
Daniel Capps

\noindent
\subsubsection{Citation}

\Urlmuskip=0mu plus 1mu\relax

\bibentry{bugeja2016privacy}

\subsubsection{Main Idea}

\noindent
The authors give an overview of the privacy and security challenges directed towards the smart home domain. The authors discuss different constraints, soulution, challenges, and research issues where further investigation is required.

\subsubsection{Theory}

\noindent
Game Theory. The authors identify different attack vectors for smart homes and give some solutions to these attacks.

\subsubsection{Method}

\noindent
The authors identified different categories of security risks and then listed what a solution would have to look like in order to mitigate the security risks.

\subsubsection{Findings}

\noindent
The authors identified different security risks involved in smart homes. For the devices in smart homes they found problems with resource constraints, lack of user interface, and a need for tamper resistant devices and data. For the communication used in smart homes they found problems with heterogeneous protocols and devices going in and out of the network. The authors also found an issue with devices in smart homes not supporting dynamic patches for various reasons. 

\subsubsection{Future Directions}

\noindent
The authors offer multiple future research directions. They recommend further investigation into, identity management, risk assessment methods, information flow control, and security management methods relating to smart home environments. 

\Urlmuskip=0mu plus 1mu\relax

\noindent
\subsection{An Active Man-in-the-Middle Attack on Bluetooth Devices}

\subsubsection{Group Member}

\noindent
Connor Leavesley

\noindent
\subsubsection{Citation}

\Urlmuskip=0mu plus 1mu\relax
%remove % to show citation, change the name to your entry
\bibentry{BluetoothMITMTal}

\subsubsection{Main Idea}

\noindent
The author looks to outline the main issues with BLE security. He focus on a possible architecture for a man-in-the-middle attack and a case study on a MITM'd device and an associated application. 

\subsubsection{Theory}

\noindent
The author is utilizing Game Theory. He is attacking a Bluetooth connection via a MITM. 

\subsubsection{Method}

\noindent
The author uses a Dax-Hub SW-28 Smart Bracelet and its associated app PowerSensor and the subject of the MITM attack. He used Kali Linux as the attacker and installed GATTacker and BtleJuice. He connected a CSR 4.0 dongle to a VM to transmit data between the app, the bracelet, and the middleman. He then uses GATTacker to intercept the BLE advertisements and copy the GATT profile of the bracelet. The author uses the GATT to simulate the device. Tal was then able to send fake to the device. 

\subsubsection{Findings}

\noindent
The author found that it was far to easy to intercept device communications and take control. Due to the inherent insecurities of the BLE protocol, he was able to change data that the device displayed, as well as take control of mobile camera via the app. 

\subsubsection{Future Directions}

\noindent
The author does not give any ideas as to future directions. The security issues with BLE are widely know, and Tal hopes to explain the security issues here and how to exploit them in a unique manner. 

\Urlmuskip=0mu plus 1mu\relax

\noindent
\subsection{{B}luetooth: {W}ith {L}ow {E}nergy comes {L}ow {S}ecurity}

\subsubsection{Group Member}

\noindent
Joshua Niemann

\noindent
\subsubsection{Citation}

\Urlmuskip=0mu plus 1mu\relax
%remove % to show citation, change the name to your entry
\bibentry{179196}

\subsubsection{Main Idea}

\noindent
The author of this paper breaks almost all the security measures relating to Bluetooth Low Energy after key generation, and builds an open-source tool that allows for the exposure of the Long-Term Key from a BLE packet if provided the pairing sequence.  The author also breaks the channel hopping algorithm, allowing for passive attackers to follow Bluetooth packets on air.

\subsubsection{Theory}

\noindent
The author directly attacks the Bluetooth Low Energy protocol and exposes weaknesses in the key generation algorithm, the channel hopping algorithm, and other flaws in the Bluetooth Low Energy protocol.  Directly attacking the protocol allows this article to resemble Game Theory.

\subsubsection{Method}

\noindent
The authors work backward from the Bluetooth protocol, breaking the security measures of Low Energy.  An Ubertooth One is used as the intercept device.  Using this ubertooth, the authors figured out how to calculate the Hop Interval, the Hop Increment, the Access Address, and CRC init.  Finally, the author works on breaking the key exchange algorithm.

\subsubsection{Findings}

\noindent
The author found that an attacker could both follow a Bluetooth packet through the channel hopping algorithm, which would allow any passive attacker to follow a Bluetooth Stream with readily-available hardware like an Ubertooth.  The author also found that the key exchange algorithm is extremely broken and can be reversed in under 1 second on a modern computer.

\subsubsection{Future Directions}

\noindent
The author outlined several next steps.  The first of which details a hypothetical attack that would allow an attacker to force key negotiation, which would use the Bluetooth Low Energy key exchange protocol, of which was already broken in this paper.  From there, an attacker could effectively decode any live Bluetooth Low Energy communication, even if not present during the initial pairing process.

\Urlmuskip=0mu plus 1mu\relax

\noindent
\subsection{{A}nalysis of {B}luetooth {T}hreats and v4.0 {S}ecurity {F}eatures}

\subsubsection{Group Member}

\noindent
Jacob Ruud

\noindent
\subsubsection{Citation}

\Urlmuskip=0mu plus 1mu\relax
%remove % to show citation, change the name to your entry
\bibentry{sandaya_sumithra_devi_2012}

\subsubsection{Main Idea}

\noindent
The authors aim to assess the security features added to bluetooth in the 4.0 security update. They specifically focus on the addition of new security association models as well as secure simple paring both in normal and low energy mode.

\subsubsection{Theory}

\noindent
This paper covers multiple scenarios that can be attributed to the standard alice, bob, and oscar method of describing game theory. In which oscar attemps to steal sensitive information from the communication between alice and bob treating the leak of any such data as a win.

\subsubsection{Method}

\noindent
The authors break down the protocol based on a framework called "A Bluetooth Threat Taxonomy" previously developed by a man by the name of John Paul Dunning. They use this framework to provide a comprehensive risk asssessment of the bluetooth 4.0 protocol and determine what likely attack vectors are. The Attack Classifications Include: Surveillance, Range Extension, Obfuscation, Fuzzer, Sniffing, Denial of Service, Malware, Unauthorized Direct Data Access, and Man In the Middle.

Upon furhter reading, the authors dive deeper into the secure simple pairing function to assess its four association models: Numeric Comparison, Just Works, Out of Band, and Passkey Entry. Each one of these models is used independantly of each other depending of the IO capabilities of the two connecting devices.

\subsubsection{Findings}

\noindent
The findings of this paper demonstrate the advantages of Bluetooth Low Energy over standard Bluetooth communication. The main reason behind this destinction lies in the fact that BLE does not use DHKE in it's exchange of authentication data. This means that the protocol is less likely to be compromised due to the fact that potential threat actors are able to compute the shared DH Key and therefore have less trouble bypassing the other security mechanisms in place. Therefore, Bluetooth LE is not affected by PFS, KCI, or MitM attacks.

\subsubsection{Future Directions}

\noindent
The authors vaugely recommend at the end of their paper that "version 4.0 undergo a continual security analysis process by people involved." They suggest the possibility of integrated security to help "protect data privacy and to prevent misuse of data."

\Urlmuskip=0mu plus 1mu\relax

\noindent
\subsection{{R}eview of the {O}pen {S}upervised {D}evice {P}rotocol ({OSDP}) for {D}o{D} {A}pplicability}

\subsubsection{Group Member}

\noindent
Quintin Walters

\noindent
\subsubsection{Citation}

\Urlmuskip=0mu plus 1mu\relax
%remove % to show citation, change the name to your entry
\bibentry{OSDPseiwgDOD}

\subsubsection{Main Idea}

\noindent
The authors review the Open Supervised Device Protocol, how it wworks, what it supports, the messages sent, and the legal implications of using the protocol in DoD facilities.  They go in depth on the specific functioning of the protocol, the messages it sends, and the wiring necessary for it.  The authors also compare it to the Wiegand protocol, covering the differences between them and the possible issues with upgrading from Wiegand to OSDP.

\subsubsection{Theory}

\noindent
The authors implement Systems theory in this article.  They observe how OSDP interfaces with other systems and how it has been implemented without attempting to attack it in any way.

\subsubsection{Method}

\noindent
They authors surveyed multiple suppliers of OSDP compatible devices and they reviewed numerous papers on OSDP.  There was no hands on research done.

\subsubsection{Findings}

\noindent
They find that OSDP is "not yet widely adopted" and it is not explicitly called for in various legal compliances.  They also found that vendors have managed to keep Wiegand relevant but it is not able to support high levels of assurance.  The authors claim that vendors use "FICAM as a selling point more than OSDP" but if OSDP is associated with FICAM it will be implemented more often.

\subsubsection{Future Directions}

\noindent
The authors indirectly recommend to "more clrealy associate" OSDP with FICAM guidance to increase the usage of it.

\Urlmuskip=0mu plus 1mu\relax
\pagebreak
