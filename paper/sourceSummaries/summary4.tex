\section{Summary 4}

\noindent
\subsection{Title}

\subsubsection{Group Member}

\noindent
Daniel Capps

\noindent
\subsubsection{Citation}

\Urlmuskip=0mu plus 1mu\relax
%remove % to show citation, change the name to your entry
%\bibentry{filizzola2018security}

\subsubsection{Main Idea}

\noindent
Main Idea

\subsubsection{Theory}

\noindent
Theory

\subsubsection{Method}

\noindent
Method

\subsubsection{Findings}

\noindent
Findings

\subsubsection{Future Directions}

\noindent
Future Directions 

\Urlmuskip=0mu plus 1mu\relax

\noindent
\subsection{An Active Man-in-the-Middle Attack on Bluetooth Devices}

\subsubsection{Group Member}

\noindent
Connor Leavesley

\noindent
\subsubsection{Citation}

\Urlmuskip=0mu plus 1mu\relax
%remove % to show citation, change the name to your entry
\bibentry{BluetoothMITMTal}

\subsubsection{Main Idea}

\noindent
The author looks to outline the main issues with BLE security. He focus on a possible architecture for a man-in-the-middle attack and a case study on a MITM'd device and an associated application. 

\subsubsection{Theory}

\noindent
The author is utilizing Game Theory. He is attacking a Bluetooth connection via a MITM. 

\subsubsection{Method}

\noindent
The author uses a Dax-Hub SW-28 Smart Bracelet and its associated app PowerSensor and the subject of the MITM attack. He used Kali Linux as the attacker and installed GATTacker and BtleJuice. He connected a CSR 4.0 dongle to a VM to transmit data between the app, the bracelet, and the middleman. He then uses GATTacker to intercept the BLE advertisements and copy the GATT profile of the bracelet. The author uses the GATT to simulate the device. Tal was then able to send fake to the device. 

\subsubsection{Findings}

\noindent
The author found that it was far to easy to intercept device communications and take control. Due to the inherent insecurities of the BLE protocol, he was able to change data that the device displayed, as well as take control of mobile camera via the app. 

\subsubsection{Future Directions}

\noindent
The author does not give any ideas as to future directions. The security issues with BLE are widely know, and Tal hopes to explain the security issues here and how to exploit them in a unique manner. 

\Urlmuskip=0mu plus 1mu\relax

\noindent
\subsection{{B}luetooth: {W}ith {L}ow {E}nergy comes {L}ow {S}ecurity}

\subsubsection{Group Member}

\noindent
Joshua Niemann

\noindent
\subsubsection{Citation}

\Urlmuskip=0mu plus 1mu\relax
%remove % to show citation, change the name to your entry
\bibentry{179196}

\subsubsection{Main Idea}

\noindent
The author of this paper breaks almost all the security measures relating to Bluetooth Low Energy after key generation, and builds an open-source tool that allows for the exposure of the Long-Term Key from a BLE packet if provided the pairing sequence.  The author also breaks the channel hopping algorithm, allowing for passive attackers to follow Bluetooth packets on air.

\subsubsection{Theory}

\noindent
The author directly attacks the Bluetooth Low Energy protocol and exposes weaknesses in the key generation algorithm, the channel hopping algorithm, and other flaws in the Bluetooth Low Energy protocol.  Directly attacking the protocol allows this article to resemble Game Theory.

\subsubsection{Method}

\noindent
The authors work backward from the Bluetooth protocol, breaking the security measures of Low Energy.  An Ubertooth One is used as the intercept device.  Using this ubertooth, the authors figured out how to calculate the Hop Interval, the Hop Increment, the Access Address, and CRC init.  Finally, the author works on breaking the key exchange algorithm.

\subsubsection{Findings}

\noindent
The author found that an attacker could both follow a Bluetooth packet through the channel hopping algorithm, which would allow any passive attacker to follow a Bluetooth Stream with readily-available hardware like an Ubertooth.  The author also found that the key exchange algorithm is extremely broken and can be reversed in under 1 second on a modern computer.

\subsubsection{Future Directions}

\noindent
The author outlined several next steps.  The first of which details a hypothetical attack that would allow an attacker to force key negotiation, which would use the Bluetooth Low Energy key exchange protocol, of which was already broken in this paper.  From there, an attacker could effectively decode any live Bluetooth Low Energy communication, even if not present during the initial pairing process.

\Urlmuskip=0mu plus 1mu\relax

\noindent
\subsection{Title}

\subsubsection{Group Member}

\noindent
Jacob Ruud
\noindent
\subsubsection{Citation}

\Urlmuskip=0mu plus 1mu\relax
%remove % to show citation, change the name to your entry
%\bibentry{filizzola2018security}

\subsubsection{Main Idea}

\noindent
Main Idea

\subsubsection{Theory}

\noindent
Theory

\subsubsection{Method}

\noindent
Method

\subsubsection{Findings}

\noindent
Findings

\subsubsection{Future Directions}

\noindent
Future Directions 

\Urlmuskip=0mu plus 1mu\relax

\noindent
\subsection{Title}

\subsubsection{Group Member}

\noindent
Quintin Walters

\noindent
\subsubsection{Citation}

\Urlmuskip=0mu plus 1mu\relax
%remove % to show citation, change the name to your entry
%\bibentry{filizzola2018security}

\subsubsection{Main Idea}

\noindent
Main Idea

\subsubsection{Theory}

\noindent
Theory

\subsubsection{Method}

\noindent
Method

\subsubsection{Findings}

\noindent
Findings

\subsubsection{Future Directions}

\noindent
Future Directions 

\Urlmuskip=0mu plus 1mu\relax
\pagebreak
