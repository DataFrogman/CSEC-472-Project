\section{Summary 7}

\noindent
\subsection{Title}

\subsubsection{Group Member}

\noindent
Daniel Capps

\noindent
\subsubsection{Citation}

\Urlmuskip=0mu plus 1mu\relax
%remove % to show citation, change the name to your entry
%\bibentry{filizzola2018security}

\subsubsection{Main Idea}

\noindent
Main Idea

\subsubsection{Theory}

\noindent
Theory

\subsubsection{Method}

\noindent
Method

\subsubsection{Findings}

\noindent
Findings

\subsubsection{Future Directions}

\noindent
Future Directions 

\Urlmuskip=0mu plus 1mu\relax

\noindent
\subsection{Extracting the Security Features Implemented in a Bluetooth LE Connection}

\subsubsection{Group Member}

\noindent
Connor Leavesley

\noindent
\subsubsection{Citation}

\Urlmuskip=0mu plus 1mu\relax
%remove % to show citation, change the name to your entry
\bibentry{zayas18}

\subsubsection{Main Idea}

\noindent
Many of the IoT devices adopting Bluetooth do not impliment the security features that Bluetooth Special Interest Group developed. To promote visability of this issue, the authors have developed an application to extract the security features of a Bluetooth device to test the security of devices. 

\subsubsection{Theory}

\noindent
The authors are utilizing Systems Theory. They are observing how a device interacts with other Bluetooth devices trying to connect to it.

\subsubsection{Method}

\noindent
The authors developed an Android application for two smartphones, the Zenfone Max 3 and the OnePlus 6. They are using  Lenovo HW02 fitness tracker as the second bluetooth device. The app extracts logs from the btsnoop log created by Android devices. The app can discover security information about the connection from the logs.

\subsubsection{Findings}

\noindent
When pairing the tracker to the phones the fitness tracker sent the Long Term Key (LTK) to the tracker with no encryption, even though both are using Bluetooth 4.2. When using the app, this was not the case. Unfortunately the authors did not go into greater detail about what security features were found, only focusing on the inital transfer of the LTK. 

\subsubsection{Future Directions}

\noindent
The authors believe that future research should be done in further improving the security of BLE, as it is here to stay. 

\Urlmuskip=0mu plus 1mu\relax

\noindent
\subsection{{O}n {B}ad {R}andomness and {C}loning of {C}ontactless {P}ayment and {B}uilding {S}mart {C}ards}

\subsubsection{Group Member}

\noindent
Joshua Niemann

\noindent
\subsubsection{Citation}

\Urlmuskip=0mu plus 1mu\relax
%remove % to show citation, change the name to your entry
\bibentry{6565237}

\subsubsection{Main Idea}

\noindent
The authors describe the problems with the Mifare Classic RFID card's random number generator and examine the real life implications of bad key generation.

\subsubsection{Theory}

\noindent
Game Theory.  The authors explore the possibility of directly attacking MiFare cards, through the Random Number Generator and otherwise.

\subsubsection{Method}

\noindent
The authors review recent revelations about MiFare random number generation and describe the problem with entropy.  They then review the probability distribution of the random number generators for Mifare cards.

\subsubsection{Findings}

\noindent
The authors found that random number generators in small cards like the Mifare are often not secure.  They found that using problems in random number generators allow for very easily breaking security in many smart cards.

\subsubsection{Future Directions}

\noindent
The authors do not describe any formal future directions, but one obvious future direction would be doing research on more vendors of smart cards.

\Urlmuskip=0mu plus 1mu\relax

\noindent
\subsection{C}onfidence in {S}mart {T}oken {P}roximity: {R}elay {A}ttacks {R}evisited

\subsubsection{Group Member}

\noindent
Jacob Ruud

\noindent
\subsubsection{Citation}

\Urlmuskip=0mu plus 1mu\relax
%remove % to show citation, change the name to your entry
\bibentry{hancke2009confidence}

\subsubsection{Main Idea}

\noindent
The authors revisit the fesibility of implementing both passive and active relay attacks against smart tokens. They also discuss the security implications should the attackers succeed. Finally, the authors discuss possible actions that device owners could take to mitigate the risk of these attacks.

\subsubsection{Theory}

\noindent
The authors use game theory to explain the red vs blue scenario, treating a win for red as a breach of information and a win for blue as witholding access control.

\subsubsection{Method}

\noindent
The authors utilize a proxy-token and proxy-reader to create a virtual clone of the authenticated users access card and relay it to the reader. This can be done by creating your own hardware or using existing tools. The authors chose to implement their own proxy reader and proxy token for this experiment.

\subsubsection{Findings}

\noindent
the authors were able to successfully perfom a relay attack against an ISO 14443A contactless system using guidelines in public literature and easily obtainable hardware. They also found that timing constraints have little to no effect against relay attacks, and although 2FA is effective if nullifies some of the advantages of smart token systems. The authors suggest using distance bounding or monitoring using a trusted interface as more effective methods against this attack, although they warn that these methods are generally more expensive or complex. 

\subsubsection{Future Directions}

\noindent
The authors leave open the option for further research on methods against relay attacks since they believe that the methods they proposed could be improved upon.

\Urlmuskip=0mu plus 1mu\relax

\noindent
\subsection{Title}

\subsubsection{Group Member}

\noindent
Quintin Walters

\noindent
\subsubsection{Citation}

\Urlmuskip=0mu plus 1mu\relax
%remove % to show citation, change the name to your entry
%\bibentry{filizzola2018security}

\subsubsection{Main Idea}

\noindent
Main Idea

\subsubsection{Theory}

\noindent
Theory

\subsubsection{Method}

\noindent
Method

\subsubsection{Findings}

\noindent
Findings

\subsubsection{Future Directions}

\noindent
Future Directions 

\Urlmuskip=0mu plus 1mu\relax
\pagebreak
