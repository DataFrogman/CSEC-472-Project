\section{Summary 10}

\noindent
\subsection{Title}

\subsubsection{Group Member}

\noindent
Daniel Capps

\noindent
\subsubsection{Citation}

\Urlmuskip=0mu plus 1mu\relax
%remove % to show citation, change the name to your entry
%\bibentry{filizzola2018security}

\subsubsection{Main Idea}

\noindent
Main Idea

\subsubsection{Theory}

\noindent
Theory

\subsubsection{Method}

\noindent
Method

\subsubsection{Findings}

\noindent
Findings

\subsubsection{Future Directions}

\noindent
Future Directions 

\Urlmuskip=0mu plus 1mu\relax

\noindent
\subsection{Trust and Security in RFID-Based Product Authentication System}

\subsubsection{Group Member}

\noindent
Connor Leavesley

\noindent
\subsubsection{Citation}

\Urlmuskip=0mu plus 1mu\relax
%remove % to show citation, change the name to your entry
\bibentry{lehtonen07}

\subsubsection{Main Idea}

\noindent
The authors examine the trust and security of RFID authentication systems using a chain-of-trust model. 

\subsubsection{Theory}

\noindent
The authors are using Classification Theory. A trust model is being constructed based on cases of misuse. 

\subsubsection{Method}

\noindent
The authors first agree on a definition of authentication. They use this definition to define a several types of authentication, such as location-based authentication. The authors construct the chains of trust and they look at threats and attacks. They then look at these chains for the risk of certain paths happening. 

\subsubsection{Findings}

\noindent
RFID trust chains are highly complex and there are many threats faced in each chain. These chains are an effective method of mapping out trust and the threats faced to help authentication designers better secure their RFID environment. 

\subsubsection{Future Directions}

\noindent
The authors offer no indication of a future direction of research. 

\Urlmuskip=0mu plus 1mu\relax

\noindent
\subsection{{R}everse {E}ngineering the {C}ommunications {P}rotocol of an {RFID} {P}ublic {T}ransportation {C}ard}

\subsubsection{Group Member}

\noindent
Joshua Niemann

\noindent
\subsubsection{Citation}

\Urlmuskip=0mu plus 1mu\relax
%remove % to show citation, change the name to your entry
\bibentry{7945583}

\subsubsection{Main Idea}

\noindent
The authors use a ProxMark 3 to investigate the feasibility of cloning a two different variants of public transport metro cards used in major cities.  Both public trasnport cards were from cities based in Spain.

\subsubsection{Theory}

\noindent
Game Theory.  The authors use weaknesses in smart card protocols to allow for easy cloning.

\subsubsection{Method}

\noindent
The authors use a ProxMark 3 RDV to actively scan for RFID cards using a long-range RFID antenna and a ProxMark connected to a laptop in a backpack.

\subsubsection{Findings}

\noindent
The authors did not find a major security issue in the RFID cards they were testing, but instead were concerned with the fact that all communication was done over cleartext channels.  

\subsubsection{Future Directions}

\noindent
The authors note that Near Field Communication is partially compatible, and that an Android device may add some interest to future research. 

\Urlmuskip=0mu plus 1mu\relax

\noindent
\subsection{{Experimental} {Design}}

\subsubsection{Group Member}

\noindent
Jacob Ruud

\noindent
\subsubsection{Citation}

\Urlmuskip=0mu plus 1mu\relax
%remove % to show citation, change the name to your entry
\bibentry{kirk2012experimental}

\subsubsection{Main Idea}

\noindent
Author Roger E. Kirk explains in depth the process for designing experiments that yield the most significant and meaningful results. He builds off of work done by Sir Ronald Fisher in the early 20th century, as well as touches on some items to be wary of when undertaking this process.

\subsubsection{Theory}

\noindent
This paper aims to effectively use scientific theory to aid researchers in properly exploring their research topics.

\subsubsection{Method}

\noindent
The author analyzes numerous types of experimental design with two main goals the mind. The first being to establish a causal connection between the independent and dependent variables, and the second being to extract the maximum amount of information with the minimum expenditure of resources.

\subsubsection{Findings}

\noindent
The author presents no clear comparison data as findings to his readers, and instead leaves the decision up to his readers to interperet his data however they see fit for the benefit of their experiment.

\subsubsection{Future Directions}

\noindent
In this section of his reasearch, the author suggests no future directions to continue his work.

\Urlmuskip=0mu plus 1mu\relax

\noindent
\subsection{Designing and implementing malicious hardeware}

\subsubsection{Group Member}

\noindent
Quintin Walters

\noindent
\subsubsection{Citation}

\Urlmuskip=0mu plus 1mu\relax
%remove % to show citation, change the name to your entry
\bibentry{kignDesigning08}

\subsubsection{Main Idea}

\noindent
The authors propose that attackers can design and attack malicious hardware to carry out attacks rather than attack the software.  They also propose that this attack vector is harder to detect and would bypass security implemented at the software level.

\subsubsection{Theory}

\noindent
Game Theory.  The authors propose and attempt an attack vector that would bypass the security of a system entirely.

\subsubsection{Method}

\noindent
The authors designed two different "Illinois Malicious Processors" to be placed on the victim hardware and they created three attacks to implement with this hardware.  The IMPs were made using modified Leon3 processors.  The first design attacked the memory of the victim hardware and their second was a complete hardware shim between the victim hardware and the victim software. They then developed a Privilege Escalation attack, a login backdoor, and a password stealer.

\subsubsection{Findings}

\noindent
The authors found that their attacks added less than 0.1 percent more logic gates to the base system and they propose that this would be even less on a larger, more complicated victim.  They also found that compared to the normal operations of the system a one time attack only increased the operation time by 1.34\% and a persistent attack only increased the operation time by 13.0\%.  

\subsubsection{Future Directions}

\noindent
The authors do not propose any specific future work, however, it would be beneficial to continue to develop hardware that would not be detected via current means of supply chain attack detection.  Most notably analog power analysis, chip identification, and operation timing detection.

\Urlmuskip=0mu plus 1mu\relax
\pagebreak
