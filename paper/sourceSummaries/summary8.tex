\section{Summary 8}

\noindent
\subsection{Title}

\subsubsection{Group Member}

\noindent
Daniel Capps

\noindent
\subsubsection{Citation}

\Urlmuskip=0mu plus 1mu\relax
%remove % to show citation, change the name to your entry
%\bibentry{filizzola2018security}

\subsubsection{Main Idea}

\noindent
Main Idea

\subsubsection{Theory}

\noindent
Theory

\subsubsection{Method}

\noindent
Method

\subsubsection{Findings}

\noindent
Findings

\subsubsection{Future Directions}

\noindent
Future Directions 

\Urlmuskip=0mu plus 1mu\relax

\noindent
\subsection{Title}

\subsubsection{Group Member}

\noindent
Connor Leavesley

\noindent
\subsubsection{Citation}

\Urlmuskip=0mu plus 1mu\relax
%remove % to show citation, change the name to your entry
\bibentry{rfid15}

\subsubsection{Main Idea}

\noindent
Many companies and people use RFID technology to secure things. The authors look to detail the basics of RFID technology and methods of attacking it. 

\subsubsection{Theory}

\noindent
The authors are using Game Theory. 

\subsubsection{Method}

\noindent
The authors cover the history, features, components, and standards of RFID reader technology. They then cover the basic security of these readers for authentication and encryption. Finally they cover some basic attacks against RFID readers and details some potential attack scenarios. 

\subsubsection{Findings}

\noindent
The security of existing RFID solutions is not where it should be. Due to manufacturers looking to lower costs, these devices tend to lack the security features needed. Many of the RFID attacks presented are trivial to carry out and are a significant problem. The feasibility of attacks is also dependent on the use of the device. Contactless payment cards are far easier to attack than reading specific RFID identifiers. 

\subsubsection{Future Directions}

\noindent
The authors provide no indication of a further direction of study. 

\Urlmuskip=0mu plus 1mu\relax

\noindent
\subsection{Title}

\subsubsection{Group Member}

\noindent
Joshua Niemann

\noindent
\subsubsection{Citation}

\Urlmuskip=0mu plus 1mu\relax
%remove % to show citation, change the name to your entry
%\bibentry{filizzola2018security}

\subsubsection{Main Idea}

\noindent
Main Idea

\subsubsection{Theory}

\noindent
Theory

\subsubsection{Method}

\noindent
Method

\subsubsection{Findings}

\noindent
Findings

\subsubsection{Future Directions}

\noindent
Future Directions 

\Urlmuskip=0mu plus 1mu\relax

\noindent
\subsection{{BLE} {B}roadcasting {I}mpact in a {R}eal {N}etwork {E}nvironment}

\subsubsection{Group Member}

\noindent
Jacob Ruud

\noindent
\subsubsection{Citation}

\Urlmuskip=0mu plus 1mu\relax
%remove % to show citation, change the name to your entry
\bibentry{boric2017ble}

\subsubsection{Main Idea}

\noindent
The Authors examine the traffic sent by individual BLE devices to determine the impact these devices can have on latency in a simulation production environment. They also do tests to determine whether or not broadcast advertisement length has any effect on message delivery.

\subsubsection{Theory}

\noindent
The authors use Scientific Theory to observe and explain the outcome of the experiment following numerous tests adhering to the scientific method.

\subsubsection{Method}

\noindent
A Raspberry Pi 2 Model B was used as the base testing hardware for this experiment. In order to achieve the results that they did, the authors also had to equip the device with a Bluefruit LE Sniffer and some specific Bluetooth libraries that are not supported out of the box by raspberry pi (libdbus-1-dev,  libdbus-glib-1-dev,  libglib2.0-dev,  libical-dev, libreadline-dev, libudev-dev, libusb-dev)

\subsubsection{Findings}

\noindent
The authors found that the neither the duration of advertisement nor the distance between the advertiser and the collector affect the I/O capabilities of the advertiser if it is alone on the network. However, signs of latency began to show themselves while testing a broadcast with 10 other devices on the network.

\subsubsection{Future Directions}

\noindent
The authors suggest future work would include testing advertisement capabilities of devices moving around on the network since they were only testing static location. They also suggest testing of a more device dense enviroment to see how the pattern they identified grows with more devices.

\Urlmuskip=0mu plus 1mu\relax

\noindent
\subsection{Title}

\subsubsection{Group Member}

\noindent
Quintin Walters

\noindent
\subsubsection{Citation}

\Urlmuskip=0mu plus 1mu\relax
%remove % to show citation, change the name to your entry
%\bibentry{filizzola2018security}

\subsubsection{Main Idea}

\noindent
Main Idea

\subsubsection{Theory}

\noindent
Theory

\subsubsection{Method}

\noindent
Method

\subsubsection{Findings}

\noindent
Findings

\subsubsection{Future Directions}

\noindent
Future Directions 

\Urlmuskip=0mu plus 1mu\relax
\pagebreak
