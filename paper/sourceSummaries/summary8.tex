\section{Summary 8}

\noindent
\subsection{{S}mart-lock security re-engineered using cryptography and steganography}

\subsubsection{Group Member}

\noindent
Daniel Capps

\noindent
\subsubsection{Citation}

\Urlmuskip=0mu plus 1mu\relax

\bibentry{bapat2017smart}

\subsubsection{Main Idea}

\noindent
The authors attempt to analyze the MITM vulnerability of BLE and develop a possible solution for designing smart-locks with an increased level of security. The authors show that using image steganography and cryptography together helps mitigate vulnerabilities of BLE protocol.




\subsubsection{Theory}

\noindent
Game Theory. The authors propose mitigation techniques for MITM attacks in BLE.


\subsubsection{Method}

\noindent
The authors propose a solution based on using the advantages of both steganography and cryptography which will increase the level of security. The techniques of cryptography and steganography are used to provide security to smart locks. A user first enters the passkey via the android app. Later, an image is selected for the passkey to be embedded into. Using AES encryption, the passkey is encrypted and then encoded in the image which all happens on the client side. The image is sent over BLE 4.0 to the Raspberry Pi. The image is then decoded and decrypted and the passkey is recovered and if it’s correct the smart lock is opened. The authors ran tests on the efficiency of the solution. 



\subsubsection{Findings}

\noindent
In the authors testing they found that there is a linear relationship between the image size and BLE transfer time which means small images transfer faster. For example, the authors found that an 6.97kb image took 19.8 seconds while an image 1100kb took 137 seconds. The authors also showed images in the paper before and after encoding and concluded that the images had no visual differences between each other.

\subsubsection{Future Directions}

\noindent
Can we make an implementation for the solution that the authors demonstrate? Is there a way to speed up the time constraint without cutting back on security? Can this solution be used to protect other systems from MITM attacks or are there any flaws with the solution shown?

\Urlmuskip=0mu plus 1mu\relax


\noindent
\subsection{Title}

\subsubsection{Group Member}

\noindent
Connor Leavesley

\noindent
\subsubsection{Citation}

\Urlmuskip=0mu plus 1mu\relax
%remove % to show citation, change the name to your entry
\bibentry{rfid15}

\subsubsection{Main Idea}

\noindent
Many companies and people use RFID technology to secure things. The authors look to detail the basics of RFID technology and methods of attacking it. 

\subsubsection{Theory}

\noindent
The authors are using Game Theory. 

\subsubsection{Method}

\noindent
The authors cover the history, features, components, and standards of RFID reader technology. They then cover the basic security of these readers for authentication and encryption. Finally they cover some basic attacks against RFID readers and details some potential attack scenarios. 

\subsubsection{Findings}

\noindent
The security of existing RFID solutions is not where it should be. Due to manufacturers looking to lower costs, these devices tend to lack the security features needed. Many of the RFID attacks presented are trivial to carry out and are a significant problem. The feasibility of attacks is also dependent on the use of the device. Contactless payment cards are far easier to attack than reading specific RFID identifiers. 

\subsubsection{Future Directions}

\noindent
The authors provide no indication of a further direction of study. 

\Urlmuskip=0mu plus 1mu\relax

\noindent
\subsection{{O}n {T}he {P}ower {O}f {A}ctive {R}elay {A}ttacks {U}sing {C}ustom-made {P}roxies}

\subsubsection{Group Member}

\noindent
Joshua Niemann

\noindent
\subsubsection{Citation}

\Urlmuskip=0mu plus 1mu\relax
%remove % to show citation, change the name to your entry
\bibentry{6810722}

\subsubsection{Main Idea}

\noindent
The authors use custom made hardware in addition to off-the-shelf smartphones to conduct an active relay attacks.

\subsubsection{Theory}

\noindent
Game theory.  The authors attack an existing security control by replaying wireless signals.

\subsubsection{Method}

\noindent
The authors researched different types of relay attacks.  They then investigated the strengths and weaknesses of using custom hardware compared to using off the shelf equipment like a smartphone.  Given the tight time constraints of the RFID standard, the authors also investigated the round trip time for different scenarios, such as Bluetooth and Wi-Fi.  The authors then investigated different restrictions in the standard, non-rooted Android operating system in terms of cloning cards.  Finally, they used this information to build a custom relay device.

\subsubsection{Findings}

\noindent
The authors found that it is possible to use multiple smartphones on a conventional channel such as Bluetooth or Wi-Fi to relay a RFID card over a long distance using some vulnerabilities in the RFID standard.  These vulnerabilities allow for the device to slow down the transaction rate using specially crafted packets. In addition, the authors built custom hardware that performed very well for these types of attacks.  The authors compared this custom hardware to that of the ProxMark, or OpenPICC RFID-hacking hardware projects.

\subsubsection{Future Directions}

\noindent
The authors did not specify any future directions.  However, a logical step here would be to look into different protocols that might theoretically allow for longer range.  For example, could the Author's solution allow for a relay attack over an LTE connection for longer distance?  Also looking at other, non-tested types of cards could be a possible future direction.

\Urlmuskip=0mu plus 1mu\relax

\noindent
\subsection{{BLE} {B}roadcasting {I}mpact in a {R}eal {N}etwork {E}nvironment}

\subsubsection{Group Member}

\noindent
Jacob Ruud

\noindent
\subsubsection{Citation}

\Urlmuskip=0mu plus 1mu\relax
%remove % to show citation, change the name to your entry
\bibentry{boric2017ble}

\subsubsection{Main Idea}

\noindent
The Authors examine the traffic sent by individual BLE devices to determine the impact these devices can have on latency in a simulation production environment. They also do tests to determine whether or not broadcast advertisement length has any effect on message delivery.

\subsubsection{Theory}

\noindent
The authors use Scientific Theory to observe and explain the outcome of the experiment following numerous tests adhering to the scientific method.

\subsubsection{Method}

\noindent
A Raspberry Pi 2 Model B was used as the base testing hardware for this experiment. In order to achieve the results that they did, the authors also had to equip the device with a Bluefruit LE Sniffer and some specific Bluetooth libraries that are not supported out of the box by raspberry pi (libdbus-1-dev,  libdbus-glib-1-dev,  libglib2.0-dev,  libical-dev, libreadline-dev, libudev-dev, libusb-dev)

\subsubsection{Findings}

\noindent
The authors found that the neither the duration of advertisement nor the distance between the advertiser and the collector affect the I/O capabilities of the advertiser if it is alone on the network. However, signs of latency began to show themselves while testing a broadcast with 10 other devices on the network.

\subsubsection{Future Directions}

\noindent
The authors suggest future work would include testing advertisement capabilities of devices moving around on the network since they were only testing static location. They also suggest testing of a more device dense enviroment to see how the pattern they identified grows with more devices.

\Urlmuskip=0mu plus 1mu\relax

\noindent
\subsection{Title}

\subsubsection{Group Member}

\noindent
Quintin Walters

\noindent
\subsubsection{Citation}

\Urlmuskip=0mu plus 1mu\relax
%remove % to show citation, change the name to your entry
%\bibentry{filizzola2018security}

\subsubsection{Main Idea}

\noindent
Main Idea

\subsubsection{Theory}

\noindent
Theory

\subsubsection{Method}

\noindent
Method

\subsubsection{Findings}

\noindent
Findings

\subsubsection{Future Directions}

\noindent
Future Directions 

\Urlmuskip=0mu plus 1mu\relax
\pagebreak
