\section{Summary 6}

\noindent
\subsection{Title}

\subsubsection{Group Member}

\noindent
Daniel Capps

\noindent
\subsubsection{Citation}

\Urlmuskip=0mu plus 1mu\relax
%remove % to show citation, change the name to your entry
%\bibentry{filizzola2018security}

\subsubsection{Main Idea}

\noindent
Main Idea

\subsubsection{Theory}

\noindent
Theory

\subsubsection{Method}

\noindent
Method

\subsubsection{Findings}

\noindent
Findings

\subsubsection{Future Directions}

\noindent
Future Directions 

\Urlmuskip=0mu plus 1mu\relax

\noindent
\subsection{Bluetooth: With Low Energy comes Low Security}

\subsubsection{Group Member}

\noindent
Connor Leavesley

\noindent
\subsubsection{Citation}

\Urlmuskip=0mu plus 1mu\relax
%remove % to show citation, change the name to your entry
\bibentry{rmike13}

\subsubsection{Main Idea}

\noindent
The simplification of Bluetooth Low Energy to reduce power use caused a number of security issues. The author gives an overview of Bluetooth Low Energy, BLE packet injection, and encryption weaknesses. 

\subsubsection{Theory}

\noindent
This paper is utilizing Game Theory. The author attacks the BLE protocol in a zero sum game. 

\subsubsection{Method}

\noindent
The author implements several types of Bluetooth attacks utilizing an Ubertooth: eavesdropping, packet injection, and bypassing encryption. He then explores the flaws and devises a number of mitigations to fix the issues present. 

\subsubsection{Findings}

\noindent
The author found that attacking BLE is trivial after a few, small technical hurdles. To eavesdrop on a BLE device, an attacker needs to know the hop interval, hop increment, access address, and CRC init. Ryan used this information to calculate which of the 37 channels the connection would switch to next. There are three types of methods to set up a Temporary Key (TK) for BLE, Just Works, with a key of all zeros, a six digit pin, or OOB, which is an agreed out-of-band 128 bit key. The author found that this key could be brute forced in less than a second if the device used a six digit pin. He can use the TK to derive the Short Term Key, and that to derive the Long Term Key. OOB is far harder for an attacker to crack, but exchanging a key out-of-band is difficult and not practical in practice. 

\subsubsection{Future Directions}

\noindent
The author believes that future researchers should investigate main-in-the-middle attacks against Bluetooth devices to increase the effectiveness of the proposed attacks. 

\Urlmuskip=0mu plus 1mu\relax

\noindent
\subsection{Title}

\subsubsection{Group Member}

\noindent
Joshua Niemann

\noindent
\subsubsection{Citation}

\Urlmuskip=0mu plus 1mu\relax
%remove % to show citation, change the name to your entry
%\bibentry{filizzola2018security}

\subsubsection{Main Idea}

\noindent
Main Idea

\subsubsection{Theory}

\noindent
Theory

\subsubsection{Method}

\noindent
Method

\subsubsection{Findings}

\noindent
Findings

\subsubsection{Future Directions}

\noindent
Future Directions 

\Urlmuskip=0mu plus 1mu\relax

\noindent
\subsection{Title}

\subsubsection{Group Member}

\noindent
Jacob Ruud

\noindent
\subsubsection{Citation}

\Urlmuskip=0mu plus 1mu\relax
%remove % to show citation, change the name to your entry
%\bibentry{filizzola2018security}

\subsubsection{Main Idea}

\noindent
Main Idea

\subsubsection{Theory}

\noindent
Theory

\subsubsection{Method}

\noindent
Method

\subsubsection{Findings}

\noindent
Findings

\subsubsection{Future Directions}

\noindent
Future Directions 

\Urlmuskip=0mu plus 1mu\relax

\noindent
\subsection{Poster: Power Replay Attack in Electronic Door Locks}

\subsubsection{Group Member}

\noindent
Quintin Walters

\noindent
\subsubsection{Citation}

\Urlmuskip=0mu plus 1mu\relax
%remove % to show citation, change the name to your entry
\bibentry{seongyeolPoster}

\subsubsection{Main Idea}

\noindent
The authors theorize that electronic door locks are susceptible to a power replay attack in which an insider modifies the lock with a bypass circuit.  This circuit can be triggered to provide power to the locking mechanism and unlock the door remotely.

\subsubsection{Theory}

\noindent
The theory shown is game theory, if the lock is susceptible to this attack then it is bypassable and considered insecure.

\subsubsection{Method}

\noindent
The authors attached a Bluetooth capable circuit and LiPo battery to the internal circuitry of the lock.  This is wired in parallel with the original locking circuit so that the lock will still perform normally while the implant is in place.

\subsubsection{Findings}

\noindent
The authors found that locks from the manufacturers Gateman, Samsung, Mille, and Hyegang are susceptible to this attack.  A malicious insider could place these implants without being detected and use it to gain access at a later date.

\subsubsection{Future Directions}

\noindent
The authors recommend adding tamper detection circuitry to the locks in order to detect if they have been opened along with additional hardware to detect changes in the internal capacitance.  These modifications would make it harder to place an implant like this and allow for an easier detection of an existing implant.

\Urlmuskip=0mu plus 1mu\relax
\pagebreak
