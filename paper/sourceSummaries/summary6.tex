\section{Summary 6}

\noindent
\subsection{{A}n efficient access control scheme for smart lock based on asynchronous communication}

\subsubsection{Group Member}

\noindent
Daniel Capps

\noindent
\subsubsection{Citation}

\Urlmuskip=0mu plus 1mu\relax

\bibentry{han2019efficient}

\subsubsection{Main Idea}

\noindent
The authors propose an asynchronous and consistent lock access management scheme to avoid unauthorized access through the lock. The authors also present a lightweight and efficient tree-based access control solution to such smart lock network’s problems, which enable cascading deletion. 

 




\subsubsection{Theory}

\noindent
Game Theory. The authors propose mitigation techniques for smart locks to defend against different exploits, primarily evasion attacks.



\subsubsection{Method}

\noindent
The authors proposed secure command transmission and a tree based access control system and then evaluated their proposed systems. The secure command transmission system is a device centered system that has two phases, the user-lock phase and the user-cloud phase. In the user-lock phase, users give an unlock request or forward the commands from the cloud to the smart lock. In the user-cloud phase, users are able to send and receive access control commands from the cloud. During this process there is a preprocessor in the smart lock that checks the security of incoming requests and determines how to handle them. The tree based access control system (TACS) is stored directly on the smart lock and every command from the owner is performed on this tree. The smart lock maintains this tree by executing commands sent by the owner. When a user’s request comes to TACS, it checks whether the user is authorized to make that request or not. 



\subsubsection{Findings}

\noindent
The authors found that against evasion attacks their scheme forces each user to connect to the cloud to get a valid period, which only then can the user unlock the door. However adversaries can still unlock the door within the valid period (authors set it to 24 hours) just after being revoked. The authors weren’t able to have all information updated immediately after an owner's command is sent due to their asynchronous communication scheme. All operations on the tree took less than 100 milliseconds in the authors experiments and in most cases the tree took up less than 1 MB and could store up to 12 nodes.

\subsubsection{Future Directions}

\noindent
Can the scheme used here for smart locks be expanded into other fields? Is there a way to remove the valid window so that adversaries can’t use the lock immediately upon being revoked?


\Urlmuskip=0mu plus 1mu\relax


\noindent
\subsection{Bluetooth: With Low Energy comes Low Security}

\subsubsection{Group Member}

\noindent
Connor Leavesley

\noindent
\subsubsection{Citation}

\Urlmuskip=0mu plus 1mu\relax
%remove % to show citation, change the name to your entry
\bibentry{mryan13}

\subsubsection{Main Idea}

\noindent
The simplification of Bluetooth Low Energy to reduce power use caused a number of security issues. The author gives an overview of Bluetooth Low Energy, BLE packet injection, and encryption weaknesses. 

\subsubsection{Theory}

\noindent
This paper is utilizing Game Theory. The author attacks the BLE protocol in a zero sum game. 

\subsubsection{Method}

\noindent
The author implements several types of Bluetooth attacks utilizing an Ubertooth: eavesdropping, packet injection, and bypassing encryption. He then explores the flaws and devises a number of mitigations to fix the issues present. 

\subsubsection{Findings}

\noindent
The author found that attacking BLE is trivial after a few, small technical hurdles. To eavesdrop on a BLE device, an attacker needs to know the hop interval, hop increment, access address, and CRC init. Ryan used this information to calculate which of the 37 channels the connection would switch to next. There are three types of methods to set up a Temporary Key (TK) for BLE, Just Works, with a key of all zeros, a six digit pin, or OOB, which is an agreed out-of-band 128 bit key. The author found that this key could be brute forced in less than a second if the device used a six digit pin. He can use the TK to derive the Short Term Key, and that to derive the Long Term Key. OOB is far harder for an attacker to crack, but exchanging a key out-of-band is difficult and not practical in practice. 

\subsubsection{Future Directions}

\noindent
The author believes that future researchers should investigate main-in-the-middle attacks against Bluetooth devices to increase the effectiveness of the proposed attacks. 

\Urlmuskip=0mu plus 1mu\relax

\noindent
\subsection{{L}ock {P}icking in the {E}ra of {I}nternet of {T}hings}

\subsubsection{Group Member}

\noindent
Joshua Niemann

\noindent
\subsubsection{Citation}

\Urlmuskip=0mu plus 1mu\relax
%remove % to show citation, change the name to your entry
\bibentry{8887393}

\subsubsection{Main Idea}

\noindent
The authors perform a security analysis of a common Bluetooth Smart Lock created by a major brand.  A MasterLock was chosen because of the strong brand recognition in the lock segment, with a smaller emphasis on IOT devices.

\subsubsection{Theory}

\noindent
Game Theory.  The authors perform an attack on a common lock in order to determine inherent weaknesses in the product.

\subsubsection{Method}

\noindent
The authors conducted an analysis on the backend server API calls.  They do this by reverse engineering the mobile application for the product.  From there, valid requests were sent to determine how the API server would handle various different scenarios.

\subsubsection{Findings}

\noindent
The authors found that guest and master codes are statically generated, meaning that they could not be revoked.  The authors also found that many APIs could be misused to generate unliimited future access codes from the perspectie of a limited guest account.

\subsubsection{Future Directions}

\noindent
The authors discuss the possibility of investigating the bluetooth connection between the lock and the phone.  Although a limited amount of bluetooth research was done in the authors' testing, a lot more work can be done in investigating the bluetooth link, particularly on guest accounts.

\Urlmuskip=0mu plus 1mu\relax

\noindent
\subsection{{I}nvestigations of {P}ower {A}nalysis {A}ttacks on {S}martcards}

\subsubsection{Group Member}

\noindent
Jacob Ruud

\noindent
\subsubsection{Citation}

\Urlmuskip=0mu plus 1mu\relax
%remove % to show citation, change the name to your entry
\bibentry{messerges_dabbish_sloan_1999}

\subsubsection{Main Idea}

\noindent
The main idea of this paper is to analyze power anylysis techniques that had been used in the past to attack the DES encryption protocol. The authors then propose a method to model the signal vs noise ratio.

\subsubsection{Theory}

\noindent
The main theory on display during this analysis is game theory, the authors treat the ability to measue power levels from the reader as a win in this scenario.

\subsubsection{Method}

\noindent
The authors were able to measure the power dissipation of the smartcard by reading from the ground pin with the assisstance of a small resistor in series between the VSS pin on the card and the true ground.

\subsubsection{Findings}

\noindent
The authors confimed "that power analysis attacks can be quite powerful and need to be addressed." They also were able to successfully propose a way to model the noise characteristics of the power signal coming from the smartcard in hopes of inspiring future work to secure smartcard software against power analysis attacks.

\subsubsection{Future Directions}

\noindent
The authors suggest that "Future research in this area will investigate power analysis attacks on hardware encryption devices and publickey cryptosystems." At the time this paper was written, AES had not yet been published so there definately room for more work on this topic.  

\Urlmuskip=0mu plus 1mu\relax

\noindent
\subsection{Poster: Power Replay Attack in Electronic Door Locks}

\subsubsection{Group Member}

\noindent
Quintin Walters

\noindent
\subsubsection{Citation}

\Urlmuskip=0mu plus 1mu\relax
%remove % to show citation, change the name to your entry
\bibentry{seongyeolPoster}

\subsubsection{Main Idea}

\noindent
The authors theorize that electronic door locks are susceptible to a power replay attack in which an insider modifies the lock with a bypass circuit.  This circuit can be triggered to provide power to the locking mechanism and unlock the door remotely.

\subsubsection{Theory}

\noindent
The theory shown is game theory, if the lock is susceptible to this attack then it is bypassable and considered insecure.

\subsubsection{Method}

\noindent
The authors attached a Bluetooth capable circuit and LiPo battery to the internal circuitry of the lock.  This is wired in parallel with the original locking circuit so that the lock will still perform normally while the implant is in place.

\subsubsection{Findings}

\noindent
The authors found that locks from the manufacturers Gateman, Samsung, Mille, and Hyegang are susceptible to this attack.  A malicious insider could place these implants without being detected and use it to gain access at a later date.

\subsubsection{Future Directions}

\noindent
The authors recommend adding tamper detection circuitry to the locks in order to detect if they have been opened along with additional hardware to detect changes in the internal capacitance.  These modifications would make it harder to place an implant like this and allow for an easier detection of an existing implant.

\Urlmuskip=0mu plus 1mu\relax
\pagebreak
