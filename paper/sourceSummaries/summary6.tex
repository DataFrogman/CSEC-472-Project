\section{Summary 6}

\noindent
\subsection{Title}

\subsubsection{Group Member}

\noindent
Daniel Capps

\noindent
\subsubsection{Citation}

\Urlmuskip=0mu plus 1mu\relax
%remove % to show citation, change the name to your entry
%\bibentry{filizzola2018security}

\subsubsection{Main Idea}

\noindent
Main Idea

\subsubsection{Theory}

\noindent
Theory

\subsubsection{Method}

\noindent
Method

\subsubsection{Findings}

\noindent
Findings

\subsubsection{Future Directions}

\noindent
Future Directions 

\Urlmuskip=0mu plus 1mu\relax

\noindent
\subsection{Bluetooth: With Low Energy comes Low Security}

\subsubsection{Group Member}

\noindent
Connor Leavesley

\noindent
\subsubsection{Citation}

\Urlmuskip=0mu plus 1mu\relax
%remove % to show citation, change the name to your entry
\bibentry{rmike13}

\subsubsection{Main Idea}

\noindent
The simplification of Bluetooth Low Energy to reduce power use caused a number of security issues. The author gives an overview of Bluetooth Low Energy, BLE packet injection, and encryption weaknesses. 

\subsubsection{Theory}

\noindent
This paper is utilizing Game Theory. The author attacks the BLE protocol in a zero sum game. 

\subsubsection{Method}

\noindent
The author implements several types of Bluetooth attacks utilizing an Ubertooth: eavesdropping, packet injection, and bypassing encryption. He then explores the flaws and devises a number of mitigations to fix the issues present. 

\subsubsection{Findings}

\noindent
The author found that attacking BLE is trivial after a few, small technical hurdles. To eavesdrop on a BLE device, an attacker needs to know the hop interval, hop increment, access address, and CRC init. Ryan used this information to calculate which of the 37 channels the connection would switch to next. There are three types of methods to set up a Temporary Key (TK) for BLE, Just Works, with a key of all zeros, a six digit pin, or OOB, which is an agreed out-of-band 128 bit key. The author found that this key could be brute forced in less than a second if the device used a six digit pin. He can use the TK to derive the Short Term Key, and that to derive the Long Term Key. OOB is far harder for an attacker to crack, but exchanging a key out-of-band is difficult and not practical in practice. 

\subsubsection{Future Directions}

\noindent
The author believes that future researchers should investigate main-in-the-middle attacks against Bluetooth devices to increase the effectiveness of the proposed attacks. 

\Urlmuskip=0mu plus 1mu\relax

\noindent
\subsection{Title}

\subsubsection{Group Member}

\noindent
Joshua Niemann

\noindent
\subsubsection{Citation}

\Urlmuskip=0mu plus 1mu\relax
%remove % to show citation, change the name to your entry
%\bibentry{filizzola2018security}

\subsubsection{Main Idea}

\noindent
Main Idea

\subsubsection{Theory}

\noindent
Theory

\subsubsection{Method}

\noindent
Method

\subsubsection{Findings}

\noindent
Findings

\subsubsection{Future Directions}

\noindent
Future Directions 

\Urlmuskip=0mu plus 1mu\relax

\noindent
\subsection{Title}

\subsubsection{Group Member}

\noindent
Jacob Ruud

\noindent
\subsubsection{Citation}

\Urlmuskip=0mu plus 1mu\relax
%remove % to show citation, change the name to your entry
%\bibentry{filizzola2018security}

\subsubsection{Main Idea}

\noindent
Main Idea

\subsubsection{Theory}

\noindent
Theory

\subsubsection{Method}

\noindent
Method

\subsubsection{Findings}

\noindent
Findings

\subsubsection{Future Directions}

\noindent
Future Directions 

\Urlmuskip=0mu plus 1mu\relax

\noindent
\subsection{Portable RFID Bumping Device}

\subsubsection{Group Member}

\noindent
Quintin Walters

\noindent
\subsubsection{Citation}

\Urlmuskip=0mu plus 1mu\relax
%remove % to show citation, change the name to your entry
\bibentry{dijkPortableBumping}

\subsubsection{Main Idea}

\noindent
The authors proposed that RFID cards may be vulnerable to a high speed cloning attack performed by bumping into the target.  They posit that this attach can be used to gain clones of security cards without the target's knowledge or raising suspicion.

\subsubsection{Theory}

\noindent
Game Theory.  By showing the ease of cloning the cards the authors show that they are broken and need increased security in order to stay useful.

\subsubsection{Method}

\noindent
The authors used a Proxmark 3, a LG Nexus 5 with Nethunter, a Hirose usb cable, MIFARE Classic 1K, and MIFARE Classic EV1 1K.  The authors wrote a default key to the cards and then a large number of random keys.  They then generated a large number of random keys to try and gathered a large number of nonces to perform their attack.

\subsubsection{Findings}

\noindent
The authors found that the HF Hirose Antenna increased the range to 6-8cm.  They also successfully implemented BTWA to read multiple cards.  The attacks against the cards were successefully performed against both the Classic and the Classic EV1.

\subsubsection{Future Directions}

\noindent
The authors posit that more research can be done into the maximum number of cards that can be read at one time.  They also believes that the attack framework can be extended to support more attacks, optimization can be performed on the keyspace calculation software, attempts can be made to try the attack in an unstable environment, and finally optimazation of the Proxmark firmware.

\Urlmuskip=0mu plus 1mu\relax
\pagebreak
