\section{Summary 5}

\noindent
\subsection{Title}

\subsubsection{Group Member}

\noindent
Daniel Capps

\noindent
\subsubsection{Citation}

\Urlmuskip=0mu plus 1mu\relax
%remove % to show citation, change the name to your entry
%\bibentry{filizzola2018security}

\subsubsection{Main Idea}

\noindent
Main Idea

\subsubsection{Theory}

\noindent
The authors demonstrate Game Theory by directly comparing different card-cloning prevention techniques and their overall effectiveness.

\subsubsection{Method}

\noindent
The authors evaluate the practicality of each detection technique or anti-cloning to determine the best option for an organization looking to secure their enviornment.

\subsubsection{Findings}

\noindent
The authors found that card cloning is hard to particularly stop.  Card cloning prevention requires a secret to be kept, and as long as the secret has to be transmitted to the reader at some point, the secret can be intercepted.  Encryption isn't often too much help either, due to the low-powered nature of the RFID cards in question.  Detection on the other hand is often a much better option.  With detection, you can probalistically determine if a user is likely a clone using enviornmental attributes as well.  This could entail detection based on a location that is improbable for a legitimate tag to be, or subsequent scans that are too far apart from each other.

\subsubsection{Future Directions}

\noindent
The authors specify no formal future directions for this research, but do mention that further research on how to detect cloning attacks is not only necesary, but required for the security of every RFID system.  The paper also mentions that more can be done on the hardware security front, such as better readers, better hash algorithms to avoid collision, strengthening the reliability of wireless connections, and better encryption techniques.

\Urlmuskip=0mu plus 1mu\relax

\noindent
\subsection{Title}

\subsubsection{Group Member}

\noindent
Connor Leavesley

\noindent
\subsubsection{Citation}

\Urlmuskip=0mu plus 1mu\relax
%remove % to show citation, change the name to your entry
%\bibentry{filizzola2018security}

\subsubsection{Main Idea}

\noindent
Main Idea

\subsubsection{Theory}

\noindent
Theory

\subsubsection{Method}

\noindent
Method

\subsubsection{Findings}

\noindent
Findings

\subsubsection{Future Directions}

\noindent
Future Directions 

\Urlmuskip=0mu plus 1mu\relax

\noindent
\subsection{{Y}ou {C}an {C}lone {B}ut {Y}ou {C}an't {H}ide: {A} {S}urvey of {C}lone {P}revention and {D}etection for {RFID}}

\subsubsection{Group Member}

\noindent
Joshua Niemann

\noindent
\subsubsection{Citation}

\Urlmuskip=0mu plus 1mu\relax
%remove % to show citation, change the name to your entry
\bibentry{7888545}

\subsubsection{Main Idea}

\noindent
The authors evaluate RFID cards and the ease of cloning them using various cloning countermeasures.

\subsubsection{Theory}

\noindent
Theory

\subsubsection{Method}

\noindent
Method

\subsubsection{Findings}

\noindent
Findings

\subsubsection{Future Directions}

\noindent
Future Directions 

\Urlmuskip=0mu plus 1mu\relax

\noindent
\subsection{Title}

\subsubsection{Group Member}

\noindent
Jacob Ruud

\noindent
\subsubsection{Citation}

\Urlmuskip=0mu plus 1mu\relax
%remove % to show citation, change the name to your entry
%\bibentry{filizzola2018security}

\subsubsection{Main Idea}

\noindent
Main Idea

\subsubsection{Theory}

\noindent
Theory

\subsubsection{Method}

\noindent
Method

\subsubsection{Findings}

\noindent
Findings

\subsubsection{Future Directions}

\noindent
Future Directions 

\Urlmuskip=0mu plus 1mu\relax

\noindent
\subsection{Security Analysis of Vendor Customized Code in Firmware of Embedded Devices}

\subsubsection{Group Member}

\noindent
Quintin Walters

\noindent
\subsubsection{Citation}

\Urlmuskip=0mu plus 1mu\relax
%remove % to show citation, change the name to your entry
\bibentry{liuSecurityAnalysis}

\subsubsection{Main Idea}

\noindent
The authors propose that the vendor customized code within device firmware is the most likely attack vector.  They cover the tools to analyze this section, methodology to test it, and then analyze five embedded devices.

\subsubsection{Theory}

\noindent
The authors implement Game Theory in this article.  They examine multiple devices for security threats and treat it as a zero sum game.

\subsubsection{Method}

\noindent
The authors assess five devices: the TP-Link WR740nv5, TOTOLINK A850R, HUAQIN HGU421, Thunder Money Maker, and Yi Smart Webcam.  They check hte device's standard functionality and then test the sections for vulnerabilities.

\subsubsection{Findings}

\noindent
The authors discover that the TP-Link WR740nv5 and the Thunder Money Maker do not perform code integrity checks.  The TOTOLINK A850R has a vulnerability in its authentication method that can cause a web server crash if an attacker sends a carefully crafted request.  Finally, the Yi Smart Webcam is vulnerable to Man-In-The-Middle attacks and the camera will respond to anyone on TCP port 38888 with the session key.  These vulnerabilities were all found in the vendor specific code for the devices, proving the author's point about the possible weakness there.

\subsubsection{Future Directions}

\noindent
The authors recommend further development in current firmware analysis tools.  Further research can be done to better them and the implementation of standards for vendor code segments in embedded devices.

\Urlmuskip=0mu plus 1mu\relax
\pagebreak